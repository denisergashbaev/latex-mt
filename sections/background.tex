\chapter{Background and definitions}
\label{sec:Background}
% neural networks 
In this section, we go over the preliminarily concepts that help understand the contributions of this thesis. We start by looking at the family of methods to which Deep Neural Networks belong to, followed by a more detailed look at the method in question and explain how we will make our contribution using Deep Neural Network. 

\section{Deep Learning}

%Today, \textit{Artificial Intelligence(AI)} is a thriving field with many practical applications and active research topics. We look to intelligent software to automate routine labor, understand speech or images, make diagnoses in medicine and support basic scientific research. 
The true challenge to Artificial Intelligence(AI) proved to be solving the tasks that are easy for people to perform but hard for people to describe formally—problems that we solve intuitively, that feel automatic, like recognizing spoken words or faces in images. The solution is to allow computers to learn from experience and understand the world in terms of a hierarchy of concepts, with each concept defined in terms of its relation to simpler concepts. The hierarchy of concepts allows the computer to learn complicated concepts by building them out of simpler ones. If we draw a graph showing how these concepts are built on top of each other, the graph is deep, with many layers. For this reason, we call this approach to AI \textit{Deep Learning}\cite{dl book}.

In other words,  Deep Learning is a new are of Machine Learning research, which has been introduced with the objective of moving ML closer to one of its original goals: AI. Deep Learning is about learning multiple levels of representation and abstraction that help to make sense of data such as images, sound and text\cite{dldef2} 

\section{Deep Neural Networks}


A standard neural network (NN) consists of many simple, connected processors called neurons, each producing a sequence of real-valued activations. Shallow NN-like models with few such stages have been around for many decades if not centuries\cite{deepnn}. However, theoretical results strongly suggest that in order to learn the kind of complicated functions that can represent high-level abstractions (e.g. in vision, language, and other AI-level tasks), one needs deep architectures. Deep architectures are composed of multiple levels of non-linear operations, such as in neural nets with many hidden layers or in complicated propositional formulate re-using many sub-formulate\cite{dl for ai}.




