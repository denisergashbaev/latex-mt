\chapter{Background and Definitions}
\label{sec:Background}
% neural networks 
In this section, we go over the preliminarily concepts that help understand the contributions of this work. We start by looking at the family of methods to which Deep Convolutional Neural Networks belong, followed by a more detailed look at the method in question \todo{better name this 'method in question'}, explaining the some hyper-parameters incorporated for optimizing the proposed model.\todo{i think the explanation of your method should be done in a diff section}

\section{Deep Learning}

%Today, \textit{Artificial Intelligence(AI)} is a thriving field with many practical applications and active research topics. We look to intelligent software to automate routine labor, understand speech or images, make diagnoses in medicine and support basic scientific research. 
One of the central challenges of Artificial Intelligence (AI) is solving the tasks that are easy for people to perform but hard for them to describe formally -- problems that we solve intuitively, that feel automatic, like recognizing spoken words or faces in images. \todo{`one approach to that challenge' maybe} The solution is to allow computers to learn from experience and understand the world in terms of a hierarchy of concepts, where each concept is defined in terms of its relation to simpler concepts. This hierarchy of concepts allows the computer to learn complex notions by building them out of simpler ones. If we draw a graph showing how these concepts are built on top of each other, it would be a deep graph with many layers. For this reason, we call this approach \textit{Deep Learning}\cite{Goodfellow-et-al-2016-Book}.

Modern deep learning provides a very powerful framework for supervised learning. By adding more layers and more units within a layer, a deep network can represent functions of increasing complexity. Most tasks that consist of mapping an input vector to an output vector, and that are easy for a person to perform quickly, can be accomplished via deep learning, given sufficiently large models and datasets of labeled training examples. Other tasks, that can not be described as associating one vector to another, or that are difficult enough such that a person would require time to think and reflect in order to accomplish the task, remain beyond the scope of deep learning for now\cite{Goodfellow-et-al-2016-Book}.

In other words,  Deep Learning is a new area of Machine Learning research, which has been introduced with the objective of moving ML closer to one of its original goals: Artificial Intelligence. Deep Learning is about learning multiple levels of representation and abstraction that help to make sense of data such as images, sound and text\cite{tutorial2014lisa}.  

\section{Deep Neural Networks}

\todo{there is a package in latex -- don't remember how it's called -- that lets you define acronynims and reuse them across the text with an automatic list of acronims being generated -- ask pablo}
A standard neural network (NN) consists of many simple, connected processors called neurons, each producing a sequence of real-valued activations. Shallow NN-like models with few such stages have been around for many decades if not centuries\cite{deepnn}. However, theoretical results strongly suggest that in order to learn the kind of complicated functions that can represent high-level abstractions (e.g. in vision, language, and other AI-level tasks), one needs deep architectures. Deep Neural Networks are composed of multiple levels of non-linear operations, such as those present in neural nets with many hidden layers or in complicated propositional formulate re-using many sub-formulate\cite{bengio2009learning}\todo{`propositional formulate re-using many sub-formulate' -- i dont really understand that}. 

%ok, now you need a transition here to other definitions

\subsection{Back Propagation}

Backward Propagation of errors (BP) was the main advance in the 1980's that led to an explosion of interest in NNs. BP is one of the most commonly used methods for training NNs. The idea behind BP is that it repeatedly adjusts the weights of the connections in the network so as to minimize a measure of the difference between the actual output vector of the network and the desired one. As a result of the weight adjustments, internal \textit{hidden} units come to represent important features of the task domain, and the regularities in the task are captured by the interactions of these units\cite{williams1986learning}.

Specifically, BP computes how fast the error changes as we adjust a hidden activity by using error derivatives with respect to hidden activities\todo{`what are hidden activities' -- it comes up out of the blue}. Since each hidden activity can have a notable effect on many output units and consequently on the error, a combination of these effects must be considered. This aggregation is done efficiently which allows us to compute error derivatives for all the hidden units quickly at the same time. Computing the error derivatives for the hidden activities, it would be easy to get the error derivatives for the weights going into a hidden unit which is the key to be able to learn efficiently. 

\subsection{Weight Sharing}
\todo{Transition missing: smth like `another itegral part part/building block/technique' of deep learning is `weight sharing'}
Weight sharing refers to having several connections controlled by a single parameter (weight). Weight sharing can be interpreted as imposing equality constraints among the connection strengths. An interesting feature of weight sharing is that it can be implemented with very little computational overhead\cite{lecun1989generalization}. The weight sharing technique has an interesting side effect of reducing the number of free parameters, thereby the capacity of the machine and improving its generalization ability\cite{lecun2010convolutional}.
\todo{how is weight sharing relevant in this research. for instance, `we will later try to modify weight sharing / capitalize on it to build a more efficient model'}

\section{Convolutional Neural Networks}

Convolutional Neural Networks are a specialized kind of neural network for processing data that has a known grid-like topology such as image data which can be thought of as a 2D grid of pixels. CNN are simply neural networks that use convolution in place of general matrix multiplication in at least one of their layers\cite{Goodfellow-et-al-2016-Book}. 
Essentially, CNNs combine three architectural ideas to ensure some degree of shift and distortion invariance of local receptive fields, shared weights (or weight replication), and, sometimes, spatial or temporal sub-sampling\cite{lecun2010convolutional} \todo{what are `local receptive fields', `spacial/temporal subsampling'}. 
The following components compose the main body of any CNN architecture:

\subsection{Convolutional layer} 

 Each unit of a convolutional layer receives inputs from a set of units located in a small neighborhood in the previous layer. With local receptive fields, neurons can extract elementary visual features such as oriented edges, end-points and corners. These features are then combined by the higher layers\cite{lecun2010convolutional}. In addition, elementary feature detectors that are useful on one part of the image are likely to be useful across the entire image. This knowledge can be applied by forcing a set of units, whose receptive fields are located at different places on the image, to have identical weight vectors\cite{williams1986learning}. The outputs of such a set of neurons constitutes a \textit{feature map}. At each position, different types of units in various feature maps compute different types of features. A sequential implementation of this, for each feature map, would be to scan the input image with a single neuron that has a local receptive field, and to store the states of this neuron at corresponding locations in the feature map\cite{lecun2010convolutional}. \\
\indent Units in a feature map are constrained to perform the same operation on different parts ot the image. A convolutional layer is usually composed of several feature maps (with different weight vectors), so that multiple features can be extracted at each location. 

\subsection{Pooling/Sub-sampling layer}
Once a feature is detected, its' exact position becomes less important as long as its' approximate position relative to other features is preserved. Furthermore, as the dimensionality of applying a filter is equal to the input dimensionality, we would not be gaining any translation invariance with these additional filters, we would be stuck doing pixel-wise analysis on increasingly abstract features. In order to solve this problem, a \textit{subsampling} layer is introduced.

 Subsampling, or down-sampling, refers to reducing the overall size of a signal. In many cases, such as audio compression for music files, subsampling is done simply for size reduction \cite{sub}. But in the domain of 2D filter outputs, subsampling  can also be thought of as reducing the sensitivity of the output to shifts and distortions. One of the most applied subsampling methods used in \cite{lecun1995comparison}, is known as `max pooling'. This involves splitting up the matrix of filter outputs into small non-overlapping grids (the larger the grid, the greater the signal reduction), and taking the maximum value in each grid as the value in the reduced matrix. By applying such a max pooling layer in between convolutional layers, we can increase spatial abstractness as we raise feature abstractness\cite{sub}.

\subsection{Activation functions}
To go from one layer to the next, a set of units compute a weighted sum of their inputs from the previous layer and pass the result through a non-linear activation function\cite{lecun2015deep}. There are many possible choices for the non-linear activation functions in a multi-layered network, and the choice of activation functions for the hidden units may often be different from that for the output units. This is a consequence of the fact the hidden and output units perform different roles\cite{bishop1995neural}. 

\indent At present, the most popular non-linear function is the Rectified Linear Units (ReLU), which is simply the half-wave rectifier $f(z) = max(z, 0)$. In the past decades, neural nets used smoother non-linearities, such as $tanh(z)$ or $1/(1+ exp(-z))$, but ReLU typically learns much faster in networks with many layers, allowing training of a deep supervised network without unsupervised pre-training\cite{lecun2015deep}. 

\indent The rectifier activation function allows a network to easily obtain sparse representations. For example, after uniform initialization of the weights, around 50\% of hidden units continuous output values are real zeros, and this fraction can easily increase with sparsity-including regularization. Apart from being more biologically plausible, sparsity also leads to mathematical advantages. On the other hand, one may hypothesize that the hard saturation at 0 may hurt optimization by blocking gradient back-propagation. However, experimental results done by\citeauthor{glorot2011deep} suggest that hard zeros can actually help supervised training\cite{glorot2011deep}.  

\subsection{Local Response Normalization}

ReLUs have the desirable property that they do not require input normalization to prevent them from saturating. If at least some training examples produce a positive input to a ReLU, learning will happen in that neuron. However, we still find that the following local normalization(LRN) scheme aids generalization. This sort of response normalization implements a form of lateral inhibition \todo{wtf is that?} inspired by the type found in real neurons, creating competition for big activities amongst neuron outputs computed using different kernels\cite{krizhevsky2012imagenet}.

\indent This scheme bears some resemblance to the local contrast normalization scheme proposed by \citeauthor{jarrett2009best} in \cite{jarrett2009best} without mean activity subtraction \todo{`mean activity subtraction' -- ??} which has led to error rate reduction in \cite{krizhevsky2012imagenet} and \cite{hinton2012improving}. 

\subsection{Fully connected/Inner product layer}

Finally, after several convolutional and max pooling layers, the high-level reasoning in the neural network is done via \textit{fully connected layers}(IP). A fully connected layer takes all neurons in the previous layer (be it fully connected, pooling, or convolutional) and connects it to every single neuron it has. Fully connected layers are not spatially located anymore (you can visualize them as one-dimensional), so there can be no convolutional layers after a fully connected layer. 




