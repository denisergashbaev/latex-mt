\newpage

\chapter{Conclusions and Future Work}
\label{sec:conclusions}

\noindent In this thesis we have made several contributions on the benefits of synthetic data generation for the application of deep convolutional neural networks for problems with small training sets casted as learning to count problems. Our contributions include 1) two synthetically generated and automatically annotated datasets for even-odd handwritten digit recognition and crowd counting in the images, 2) counting the number of pedestrians in images in a privacy-preserving manner (due to the synthetic nature of images) involving no object tracking or detection techniques, 3) deep convolutional-based architectures to sufficiently learn object representations without the need for hand-crafted feature detectors, 4) a deep architecture trained with synthetic data which is capable of counting pedestrians in a real-world problem, and 5) the experimental validation of our improvements with the previous feature detecting techniques as well as with related methods in the state-of-the-art. 

%\noindent In this thesis we have made several contributions to the object representation problem casted as learning to count tasks. Our contributions include 1) a deep algorithm to sufficiently learn object representations without the need for hand-crafted feature detectors, 2) counting the number of pedestrians in images in a privacy-preserving manner involving no object tracking or detection techniques, 3) two synthetically generated and automatically annotated datasets for even-odd hand written digit recognition and crowd counting in the images, 4) a deep architecture trained with synthetic data which is capable of counting pedestrians in a real-world problem, 5) the experimental validation of our improvements with the previous feature detecting techniques as well as with related methods in the state-of-the-art. 

\section{Conclusions}

This work started off with verifying our very first hypothesis. We hypothesized that synthetic data can be applied as a well-suited surrogate for the application of deep convolutional neural networks in problems with small datasets. We introduce tasks of counting objects as problems in which large amount of labeled data is missing or hard to collect. To this end, we introduced a synthetic set of images with up to 15 MNIST digits in the images with the number of even digits as labels (section~\ref{subsubsec:digit}). In addition, we configured a deep CNN for an even-digits counting problem. In addition, Having a deep model trained, we proved the functionality of synthetic datasets for counting objects using DCNN. Moreover, we obtained noteworthy results implying applicability of deep CNN as a surrogate for previous object detection tasks using manually designed feature detectors.


   
%This work started off with verifying our very first hypothesis, where we hypothesized that deep features learned by deep convolutional neural network can be a surrogate for object detection tasks using manually designed feature detectors. To this end, we configured a deep CNN for an even-digits counting problem. We trained the network with synthetic dataset of up to 15 MNIST digits in the images with the number of even digits as labels (section~\ref{subsubsec:digit}). Having a deep model trained, we obtained noteworthy results implying applicability of deep CNN as a surrogate for previous feature detection approaches.

\indent Furthermore, we examined our methodology on a more complex problem of counting the number of pedestrians in a walkway. Applying deep learning methods for supervised learning problem relies on availability of big annotated data. Exhaustive labeling becomes prohibitive in most of the cases. To overcome this issue, we created another synthetic dataset of pedestrians in the walkway labeled with the number of people present in the images (see chapter~\ref{subsec:synped}). Designing another deep network, once again we achieved satisfactory results while discarding extensive labeling effort and hand-crafting feature detectors.   

\indent In addition to the introduced synthetic data generation algorithms, one of our main contributions and the novelty of this work was to evaluate our proposed methodology from a practical aspect by testing it on real-world problems. For that purpose, we tested our model which was trained with highly-realistic synthetic dataset, on a similar but relatively small set of real images which were labeled manually with the number individuals in the image(see section~\ref{subsec:datareal2}).    

Comparing our model's performance with a related crowd counting study (using manual labeling and highly specialized feature descriptors) on the same dataset, we may conclude that the application of deep features, not only softens annotation efforts and omits the usage of numerous hand-crafted feature detectors, but also performs trustworthy, considering the results of such previous approaches. In addition, preserving individuals' privacy will no longer be a matter of concern since our methodology does not include any object tracking techniques and applies synthetic images.    

\noindent In conclusion, the findings of this master's thesis suggest the generation of synthetic images for deep learning techniques as a suitable replacement for the lack of training data in counting problems. In addition, this work shows the applicability of deep convolutional neural networks as a surrogate for previous object counting problems. Furthermore, it substantiates the feasibility of incorporation of such methods in real world problems by alleviating the learning process while obtaining noteworthy results.  

%\noindent In conclusion, the findings of this master's thesis suggest deep learning techniques as a well-suited surrogate for previous object counting problems by showing representativity for the object of interest in similar but different tasks. It also proposes the usage of synthetic dataset to ease the annotation effort in supervised learning approaches. Furthermore, it substantiates the feasibility of incorporation of such methods in real world problems by alleviating the learning process while improving the results.  

\section{Future Work}

Any research study opens up some interesting lines of research that deserve further attention and study. There are questions and limitations that need to be addressed along with the improvements that can be made. Our work is no exception. In this master's thesis, we have observed and pointed out shortcomings and areas in which, one can improve our work (regarding datasets (chapter~\ref{dataha}), and networks architectures (chapter~\ref{imparch})). Some of the deficits of our study along with some hints to advance and enrich this research which were out of scope of this work, are mentioned below:

\begin{enumerate}

	\item We generated synthetic gray-scale datasets to tackle our proposed problems. Although we endeavored to improve the data to make it look as realistic as possible, there are still many other ways to improve the data and undoubtedly the performance of the model. One idea could be feeding colorful images to a deeper network for learning more features and predicting more accurate forecasts.   
	
	\item Due to the intuitive nature of deep networks design, one can configure and design a model more in-tune with the problems. Making the network deeper with more parameters might be an option.
	
	\item Another idea of ours for improving the performance of our crowd counting model on the real dataset is to use transfer learning\footnote{Transfer learning is the improvement of learning by transferring the knowledge from a related learned task.}. In this case we could couple two architectures: one with the synthetic data and label and the other with the real images. In this way, we can use the features learned on the synthetic data architecture for testing the real data simultaneously and update the weights of the training architecture with synthetic samples. 
	
	\item Another analysis could be a combination of deep CNN to learn to count out of sample. For instance, our even-digits counting dataset contained up to 15 digits. However, one could train a deep network with images with up to for example 30 digits in each image and label images with the number of all digits in the image and not just even digits. We assume that a combination this model and ours could count the number of even digits in images with more than 15 even digits. The same experiment can be done for counting pedestrian task in order to provide a model capable of counting people in very crowded scenes such as demonstration, marathon, etc. 
	
	\item A combination of such deep algorithm for crowd counting can be combined with faster programs to develop real-time pedestrian detection. Due to the different size and appearance of real pedestrians, deep algorithms might not be fast enough to detect pedestrians in the moment. However, faster vision detection algorithms such as \textit{cascade detection} can be used in early stages of the system to fasten the detection  process while a deep classifier can improve the accuracy of the algorithm at the final stage. An algorithm with such optimal trade-off between detection accuracy and speed can be used in various industries such as pedestrian detection systems in self-driving vehicles. 
	
	\item Another process to obtain more realistic pedestrians would be using 3D human models in cross platforms Cinema 4D or Unity to create super-realistic set of various pedestrians with different poses and style. Generating pedestrians in this way would completely remove the halo issue (described in chapter~\ref{dataimp}) around the pedestrians. 
\end{enumerate}  


