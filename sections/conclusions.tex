\newpage

\chapter{Conclusions and Future Work}
\label{sec:conclusions}

\noindent

In this thesis we have made various contributions to the object segmentation method Cage Active Contours (CAC) originally introduced in \cite{ipcac2015}. Our contributions include 1) the introduction of two different energy functions on different color spaces which have greatly enhanced the potential of an otherwise limited method. 2) the experimental validation of our improvements with the previous energies in CAC as well as with three different related methods.  3) the formalization, in mathematical terms, of some the implications and uses of the resulting segmentation \textit{cage} (a component of CAC which is used to parametrize the contour). 4)  a highlight of the possible applications of this cage to image morphing, warping as well as in shape description. 5) the creation of a public implementation in Python (with some wrapped functions in C) of cage active contours.

\section{Conclusions}

 One of the main contributions in this work is the creation of two different energy functions. These are enhanced versions of simpler energies proposed so far in CAC which have also been extended to two different color spaces. The first one is the Multivariate Mixture Gaussian energy which is an extension to the RGB color space of the Gaussian energy defined in \cite{ipcac2015} with two more improvements: the ability to capture multiple value components in each region by using Mixture Gaussian density function, and the incorporation of an initial seed which will provide the energy with prior information about the foreground and background's distributions.

The second energy is the Mean Hue Energy, which is the analogous of the Mean Energy in \cite{ipcac2015} but on the Hue component of the HSI/HSV color spaces. Because of the Cyclic nature of the Hue component, we have had to turn to cyclic spaces to use concepts such as distance, directed distance as well as a way of finding the gradient of a of the image with respect to a control point's value.

Through quantitative and qualitative experimentation on three different datasets (section \ref{sec:experiments}), we have observed that the contribution with the most impact are the extension of the Gaussian energy to the RGB color space followed by the ability of Mixture Gaussian energies to describe a region with more than one component thus obtaining a more sophisticated representation of a region, as opposed to the polarization of pixel values in previous energies. We have also seen how the incorporation of a seed in each region does not only simplify the process but it also increases its computational speed. In the case of color images, we have seen how the Mean Hue energy, despite its simplicity, can occasionally outperform the Multivariate Mixture Gaussian energy thanks to its focus on the Hue component of the image, which reflects an intrinsic property of objects, invariant to illumination.

Furthermore, we have mathematically formalized the concepts of \textit{cage}, \textit{contour}, \textit{family of contours} and others to be able to prove that two contours are similar if their cages are similar given some intial conditions. This theoretical proof, along with the properties of mean value coordinates, used widely in computer graphic applications, have allowed us to define the conditions which allow for automatic morphing and warping between similar objects, as well provide some initial intuitions for the possibility of shape description.

Our last contribution is a public implementation of Cage Active Contours in Python with some wrappers in C. The code contains different Energy functions we have presented including the ones presented  in~\cite{ipcac2015}, as well as some tools for automatic morphing and warping. The code can be found in \url{https://github.com/Jeronics/cac-segmenter/}.


\section{Future Work}
In this work, we have also observed and pointed out the limitations that hinder the performance of the segmentation using Cage Active Contours. We have seen that CAC are not designed for high precision segmentation of arbitrary images, but rather, they provide a smooth general contour of the image which can be used for other purposes and applications. The limitations of this method can be divided into two categories: those that are dependent on energy functions and those that are dependent on the cage. 

With these considerations in mind, we have proposed a set of solutions which could be evaluated and considered for future work.

The first point to address is the cage. We have seen that the restrictions of the first stage on the vertex movements are too strong and prevent CAC from adopting complex figures. Also, they prevent the cage from rotate to better adapt to a shape. Possible solution include the weighted mean between the real direction and the projected direction or  the exploration of other internal energies that could be applied to the cage for stability.

In terms of energy functions, the most challenging task ahead, by far, is to improve the Mean Hue energy which, despite its theoretically good properties in illumination invariance, has performed poorly on the quantitative experiments on real images. However we are confident that through some improvements we could be able to rise its performance to a competent level. The first improvement we propose is the adaptation of the Gaussian Energy to this space and ultimately to the Mixture Gaussian model. This extension requires the use of analogous density functions in the cyclic domain which are known as Wrapped Density Functions. The second improvement would come from the extension of an energy to the whole HSI or HSV space. As we have concluded in section \ref{subsec:extending_to_hsi} this requires tools from directional statistics and in particular, the study of cylinder spaces which these color spaces define.

As far as the implementation goes, it would be interesting to allow for a more intuitive and user-friendly interactive interface. Some of the features this could include:
\begin{enumerate}
	\item User interaction with the cage so that control points would be able to be dragged to provide a better initialization, and to correct a segmentation at a certain point.
	\item A Morphing and Warping interface that would allow for fine-tuning of morphed objects or reassigning of correspondence in points as we have done in the morphing between the pear and apple in figure~\ref{fig:car_fruits}.
\end{enumerate} 

An interesting point we have begun to raise in this thesis is the formalization in mathematical terms of the different components in Cage Active Contours to study their properties and limitations in different applications once a segmentation has been achieved. In the process of bringing solutions, more questions are brought forward to discuss and tackle in future work. Namely: \\
Given a regular cage-configuration condition:
\begin{itemize}
	\item What is the complexity of contours that can be expressed in a contour Family?
	\item Which characteristics must a cage fulfill in order to avoid its corresponding contour to self intersect?
	\item Is there a bijective map between the equivalence classes of cages and the equivalence classes of contours in a contour Family?
\end{itemize}


