%\part{title}
\chapter{Introduction}
\label{sec:introduction}
\section{Motivation}
 
% \todo{again, it's not a `milestone' as i believe. you need a different wording here}WHAT I MEAN HERE IS NoT WHAT WE DO AS PROJECT, IS THE FUNDAMENTAL CONCEPT OF LEARNING TO COUNT, LIKE WHEN KIDS LEARN TO COUNT NUMBERS
The fundamental concept of learning to count is an important educational/developmental milestone which constitutes the most fundamental idea of mathematics. In Computer Vision \cite{umbaugh1997computer}, the counting problem is the estimation of number of objects in a still image or video frame. Learning to count visual objects is a new approach towards dealing with detecting object in the images and video, which has been recently proffered in the literature \cite{viola2005detecting, rabaud2006counting, kong2005counting, chan2008privacy, segui2015learning}. It arises in many real-world applications, including cell counting in microscopic images \cite{flaccavento2011learning}, monitoring crowds in surveillance systems \cite{rahmalan2006crowd, valera2005intelligent}, and performing wildlife census or counting the number of trees in an aerial image of a forest \cite{brandtberg1998automated, pollock1996automatic,NIPS2010_4043}. 

Artificial intelligence and computer vision share topics such as pattern recognition and machine learning techniques \cite{michalski2013machine, mitchell1997machine}. Consequently, computer vision is sometimes seen as a part of the artificial intelligence field or the computer science field in general. Recent machine learning methods applied for computer vision tasks, require large number of data for the learning process. To learn to count the object of interest in an image or video, various object features need to be designed, extracted or detected during the learning phase. The complexity of feature detection process in vision tasks, restrict their usage in large-scale computer vision applications thus demanding more efficient solutions to alleviate, expedite and improve this process. 

\indent One recent and commonly used method to facilitate feature detection process is application of deep Convolutional Neural Networks(CNN) \cite{szegedy2015going, krizhevsky2012imagenet, lecun1995convolutional, sermanet2013overfeat, ji20133d, taylor2010convolutional}. One of the promises of deep CNN is replacing handcrafted features with efficient algorithms for unsupervised or semi-supervised feature learning and hierarchical feature extraction \cite{song2013hierarchical}. CNNs have been claimed and practically proven to achieve the most assuring performance in different vision benchmark problems concerning feature detection and classification \cite{ciresan2011flexible, szegedy2015going, ciresan2012multi}. 

\section{Objectives}
This work puts forward several objectives:
\begin{enumerate}

\item To abate extensive highly-specialized feature detection efforts by substituting deep CNN for learning features in problems concerning counting object of interest where no explicit information about what we are counting is given to the system, except for the object's multiplicity in the image.

\item To prevail over exhaustive data annotation required for fully supervised learning to count tasks (when using DL algorithms), by introducing synthetically created and fully automatically annotated datasets to train deep CNN with.   

\item To explore the behavior of proposed algorithms on synthetic datasets of different types and complexity, and compare the performance with state-of-the-art outcomes \cite{segui2015learning}.   

\item To analyze the performance of designed system (trained with synthetic dataset) on real-world crowd counting problems and compare the results with previously developed systems in state-of-the-art \cite{chan2008privacy}.   
 

\end{enumerate}
\section{Contributions}
Following the course of achieving fore-mentioned objectives, this dissertation contains the following contributions:

\begin{enumerate}
	\item It proposes the problem of object representation as an indirect learning problem casted as learning to count strategy. The devised algorithm is capable of counting the number of even hand-written digits in images. Moreover, the model is able to be applied for different but related tasks such as even-odd digit recognition, to demonstrate the capability of the learned features of the network for classifying digits with no direct supervising while training. 
	
	\item It provides a deep convolutional neural network for counting the number of pedestrians in a walkway that does not depend on object detection or feature tracking. The model is privacy-preserving in a sense that instead of tracking people, it learns the individual's features.
	
	\item It provides two synthetically created datasets for the proposed counting problems. First set of images contain hand-written digits for counting the number of even digits in the images. The second dataset is consist of pedestrians in a walkway, to train a counting deep convolutional neural network which is adequate for apprehending the underlying representations.
	 

	%\item It provides a synthetically generated and automatically labeled dataset of pedestrians using unlabeled University of California San Diego(UCSD) pedestrian dataset used in \cite{mahadevan2010anomaly}, to train a counting deep convolutional neural network which is adequate for apprehending the underlying representations of the learned features. To this end, we describe a counting problem for a synthetically created dataset of hand-written digits to demonstrate the capability of the internal representation of the network for classifying digits with no direct supervising while training. 
	\item The proposed model is able to count the number of people in the real and unseen dataset using the features learned by training the network on synthetic dataset. To our knowledge, this is the first crowd counting system trained by synthetic data that successfully operates on real data. 

	\item Validation of our proposal is done in the following steps:

	\begin{itemize}
		\item First, we learn to count even hand-written digits in images. Then, we evaluate the learned features of the proposed model on an even-odd digits recognition task. 
		\item Second, we validate a more complex model  quantitatively on a large synthetic dataset of pedestrians, containing maximum 29 people in each image. 
		\item Finally, we test our model's performance by counting the number of crowd in a real-world manually labeled dataset of people present in a walkway provided by~\citealt*{chan2013ground}. 
	\end{itemize}
	
\end{enumerate}

\section{Organization}

This report takes off with a review of Deep Learning(DL) (chapter~\ref{sec:dl}) as a branch of Artificial Intelligence (AI) which deep convolutional neural networks belong to, and moves on to introduce a deep CNN's basic architecture and components in details (chapter~\ref{sec:deepcnn}). 


Chapter ~\ref{sec:stateoftheart} reviews state-of-the-art of feature detection and learning to count problems. 

In Chapter~\ref{sec:proposal}, we describe our proposal for constructing a deep neural network to tackle feature detection issue learning to count problems. 

Chapter ~\ref{sec:implementation} introduces the applied platform to implement our methodology  along with the peculiarities of proposed data creation process and network modeling. 

Chapter ~\ref{sec:experiments} deals with the empirical experiments and analysis. Obtained results and comparisons with state-of-the-art solutions are expressed in this section.

Lastly, in Chapter~\ref{sec:conclusions}, we conclude the report with a short summary of the scope of work conducted and the new areas of research that this master thesis has opened.
