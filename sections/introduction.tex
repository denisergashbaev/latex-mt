\chapter{Introduction}
\label{sec:introduction}




%2. What is the background, the context, in which the research took place? 
Learning to count is an important educational/developmental milestone which constitutes the most fundamental idea of mathematics. In Computer Vision, the counting problem is the estimation of the number of objects in a still image or video frame. Learning to count visual objects is a new approach towards dealing with detecting objects in the images and video has been recently proffered in the literature. It arises in many real-world applications including cell counting in microscopic images, monitoring crowds in surveillance systems, and performing wildlife census or counting the number of trees in an aerial image of a forest\cite{NIPS2010_4043}. 

\section{Motivations}
\section{Objectives}
\section{Contributions of the Research}
\section{Organization}
%3. Why is this subject or issue important 


%4. Who are the key participants and/or ‘actors’ in the area under investigation? 

%5. Are there important trends or pivotal variables of which the reader needs to be made aware? 

%6. A clear and succinct statement of the aims and objectives that the dissertation is going to address. 

%7. Have you presented a clear and unambiguous exposition of your research aim, the objectives you will address to meet this aim and your research questions? 

%8. The reasons why this study was carried out 

%9. Was this study undertaken for example in order to test some aspect of  professional or business practice or theory or framework of analysis? 

%10. Was the research carried out to fulfil the demands of a business  organisation? 
%11. The way the Dissertation is to be organised 


