%\part{title}
\chapter{Introduction}
\label{sec:introduction}
\section{Motivation}
%\cite{umbaugh1997computer}
Learning to count is a central concept which constitutes the most fundamental idea of mathematics. In Computer Vision (CV), the counting problem is the estimation of number of objects in a still image or video frame. Learning to count visual objects is a new approach towards dealing with detecting object in the images and video, which has been recently proffered in the literature \cite{rabaud2006counting, kong2005counting, chan2008privacy, segui2015learning}. It arises in many real-world applications, including cell counting in microscopic images \cite{flaccavento2011learning}, monitoring crowds in surveillance systems \cite{rahmalan2006crowd}, and performing wildlife census or counting the number of trees in an aerial image of a forest \cite{pollock1996automatic,NIPS2010_4043}. 

Artificial intelligence and computer vision share topics such as pattern recognition and machine learning techniques \cite{mitchell1997machine}. Consequently, computer vision is sometimes seen as a part of the artificial intelligence field or the computer science field in general. Recent machine learning methods applied for computer vision tasks, require large number of data for the learning process. In order to learn to count the object of interest in an image or video, various object features need to be designed, extracted or detected during the learning phase. The complexity of feature detection process in vision tasks, restricts their usage in large-scale computer vision applications thus demanding more efficient solutions to alleviate, expedite and improve this process. 

\indent One of the recent and commonly used methods to facilitate feature detection process is application of deep convolutional neural networks \cite{szegedy2015going, krizhevsky2012imagenet, lecun1995convolutional}. One of the promises of DCNN is replacing handcrafted features with efficient algorithms for unsupervised or semi-supervised feature learning and hierarchical feature extraction \cite{song2013hierarchical}. CNNs have been claimed and practically proven to achieve the most assuring performance in different vision benchmark problems concerning feature detection and classification \cite{ciresan2011flexible, szegedy2015going, ciresan2012multi}. 

Although access to fast computers and vast amounts of data has enabled the advances of deep learning algorithms such as CNN in solving many problems that were not solvable using classic AI\footnote{Computer programs designed to solve problems that human brains performed easily, such as understanding text or recognizing objects in an image. Due to their poor performance, they introduced \textit{Expert Systems} (computer programs combined with rules provided by domain experts) which were highly tuned for a specific problem. }, they have limitations. For instance, they don't perform well when there is limited data \cite{griffin2007caltech}. This constrain restricts the application of DL methods in various areas including computer vision where they have shown promising performances.  


\section{Objectives}
This work puts forward several objectives:
\begin{enumerate}

\item To prevail over the main limitation of deep learning regarding problems with small datasets, by introducing the benefit of synthetic data generation as a surrogate for needed large training sets in fully supervised learning to count tasks in computer vision (when using DL algorithms) where DL methods haven't been applied yet due to the lack of annotated data.

\item To abate extensive highly-specialized feature detection efforts, by substituting deep CNN for learning features in problems concerning counting object of interest, where no explicit information about what we are counting is given to the system, except for the object's multiplicity in the image.
%\todo{this first objective just too long and not readable. abate is is too fancy and i think not appropriate in the context}

%\item To prevail over exhaustive data annotation required for fully supervised learning to count tasks (when using DL algorithms), by introducing synthetically created and fully automatically annotated datasets for training deep CNNs.   

\item To study and analyze the process of generating synthetic images by exploring the behavior of proposed algorithms on synthetic datasets of different types and complexity, and compare the performance with state-of-the-art outcomes \cite{segui2015learning}. 

%\item To explore the behavior of proposed algorithms on synthetic datasets of different types and complexity, and compare the performance with state-of-the-art outcomes \cite{segui2015learning}.   

\item To analyze the performance of designed system (trained with synthetic dataset) on real-world crowd counting problems and compare the results with previously developed state-of-the-art systems \cite{chan2008privacy}.   
 

\end{enumerate}
\section{Contributions}
Following the course of achieving aforementioned objectives, we created and designed different synthetic datasets and algorithms to provide an exhaustive analysis. Hence, this dissertation contains the following contributions: 
%\todo{really? please improve it -- fore-mentioned  doesn't even exist as a word. More of a transition here needed... how you've achieved that, what you've desined...}

\begin{enumerate}
	
	\item It provides a thorough study on the procedure of generating synthetic datasets for learning to count the number of objects in images by the means of deep convolutional neural networks. In this course, we present two synthetically created datasets. The first set of images contains hand-written digits for counting the number of even digits in the images. The second dataset consists of pedestrians in a walkway, to train a counting deep convolutional neural network which is adequate for apprehending the underlying representations.
	
	
	\item It proposes the problem of object representation as an indirect learning problem casted as learning to count strategy. The devised algorithm is capable of counting the number of even hand-written digits in images. Moreover, the model is able to be applied for different but related tasks such as even-odd digit recognition, to demonstrate the capability of the learned features of the network for classifying digits with no direct supervision while training. 
	
	%\todo{who `it'? THE DISSERTATION, why provides and not proposes/devises, suggests an architecture...}
	\item It suggests a deep convolutional neural network for counting the number of pedestrians in a walkway that does not depend on object detection or feature tracking. The model is privacy-preserving in a sense that instead of tracking people, it learns the individual's features.
	%\todo{`it provides' again?? CHILL BITCH SORRY MY BAD XD}
	%\item It provides two synthetically created datasets for the proposed counting problems. The first set of images contains hand-written digits for counting the number of even digits in the images. The second dataset consists of pedestrians in a walkway, to train a counting deep convolutional neural network which is adequate for apprehending the underlying representations.
	 

	%\item It provides a synthetically generated and automatically labeled dataset of pedestrians using unlabeled University of California San Diego(UCSD) pedestrian dataset used in \cite{mahadevan2010anomaly}, to train a counting deep convolutional neural network which is adequate for apprehending the underlying representations of the learned features. To this end, we describe a counting problem for a synthetically created dataset of hand-written digits to demonstrate the capability of the internal representation of the network for classifying digits with no direct supervising while training. 
	\item The proposed model is able to count the number of people in the real and unseen dataset using the features learned by training the network on synthetic dataset. To our knowledge, this is the first crowd counting system trained by synthetic data that successfully operates on real data. 

	%\todo{you are not a robot. put more sentences in here... like we devised a new validation system since none exists blah-blah. does it even belong in here?}
	\item Finally, we evaluate our proposal in different aspects regarding our objectives. In addition, fair comparisons between our methodology and different state-of-the-arts will demonstrate our improvements. In general, the validation of our proposal is done in the following steps:

	\begin{itemize}
		\item First, we learn to count even hand-written digits in images. Then, we evaluate the learned features of the proposed model on an even-odd digits recognition task. 
		\item Second, we validate a more complex model  quantitatively on a large synthetic dataset of pedestrians, containing maximum 29 people in each image. 
		\item Finally, we test our model's performance by counting the number of crowd in a real-world manually labeled dataset of people present in a walkway provided by~\citealt*{chan2013ground}. 
	\end{itemize}
	
\end{enumerate}

\section{Organization}

This report takes off with a review of deep learning (chapter~\ref{sec:dl}) as a branch of artificial intelligence which deep convolutional neural networks belong to, and moves on to introduce a deep CNN's basic architecture and components in details (chapter~\ref{sec:deepcnn}). 


Chapter ~\ref{sec:stateoftheart} reviews state-of-the-art of synthetic data generation in CV along with feature detection and learning to count problems. 

In Chapter~\ref{sec:proposal}, we introduce our created synthetic datasets to be applied in our proposal for constructing a deep neural network to tackle feature detection issue learning to count problems. 

Chapter ~\ref{sec:implementation} introduces the applied platform to implement our methodology along with the peculiarities of proposed data creation process and network modeling. 

Chapter ~\ref{sec:experiments} deals with the empirical experiments and analysis. Obtained results and comparisons with state-of-the-art solutions are expressed in this section.

Lastly, in Chapter~\ref{sec:conclusions}, we conclude the report with a short summary of the scope of work conducted and the new areas of research that this master thesis has opened.
