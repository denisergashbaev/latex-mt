\chapter{Introduction}
\label{sec:introduction}




%2. What is the background, the context, in which the research took place? 
Learning to count is an important \todo{"educational/developmental milestone" --- hmm, it's smth different. maybe concept or ability} educational/developmental milestone  which constitutes the most fundamental idea of mathematics. In Computer Vision, the counting problem is the estimation of the number of objects in a still image or video frame. Learning to count visual objects is a new approach towards dealing with detecting objects in the images and video, which has been recently \todo{proffered means offered, do you mean that?} proffered in the literature\todo{such as paper1, paper2,...}. It arises in many real-world applications, including cell counting in microscopic images, monitoring crowds in surveillance systems, and performing wildlife census or counting the number of trees in an aerial image of a forest\cite{NIPS2010_4043}\todo{citation missing}. 

\section{Motivations}
\section{Objectives}
\section{Contributions of the Research}
This thesis contains the following contributions:
\begin{enumerate}
	\item It proposes the problem of object representation as an indirect learning problem casted as learning to count strategy. The devised algorithm is capable of counting the number of pedestrians in the image that does not depend on object detection or feature tracking. The model is also privacy-preserving in a sense that it can be implemented with hardware that does not produce a visual record of the individuals in the scene. 
	\item It provides a synthetically generated and automatically labeled dataset of pedestrians using unlabeled University of California San Diego(UCSD) pedestrian dataset used in \cite{mahadevan2010anomaly}, to train a counting deep convolutional neural network which is adequate for apprehending the underlying representations of the learned features. To this end, we describe a counting problem for MNIST dataset to demonstrate the capability of the internal representation of the network for classifying digits with no direct supervising while training. 
	\item The proposed model is able to count the number of people in the real and unseen dataset using the features learned by training the network on synthetic training set. To our knowledge, this is the first crowd counting system trained by synthetic data that successfully operates continuously on real data. 
	\item Along with the validation of our proposal in the following ways:
	\begin{itemize}
		\item First, we learn to count even hand-written digits using MNIST dataset. 
		\item Second, we validate the system quantitatively on a large synthetic dataset of pedestrian, containing 100,000 images with maximum 30 pedestrians in each image. 
		\item Last but not least, we count the number of pedestrians in the manually labeled dataset of 3375 images provided by[\citeauthor*{chan2013ground}, \citeyear{chan2013ground}]. 
	\end{itemize}
	
\end{enumerate}

\section{Organization}
%3. Why is this subject or issue important 


%4. Who are the key participants and/or ‘actors’ in the area under investigation? 

%5. Are there important trends or pivotal variables of which the reader needs to be made aware? 

%6. A clear and succinct statement of the aims and objectives that the dissertation is going to address. 

%7. Have you presented a clear and unambiguous exposition of your research aim, the objectives you will address to meet this aim and your research questions? 

%8. The reasons why this study was carried out 

%9. Was this study undertaken for example in order to test some aspect of  professional or business practice or theory or framework of analysis? 

%10. Was the research carried out to fulfil the demands of a business  organisation? 
%11. The way the Dissertation is to be organised 


