\part{title}\chapter{Introduction}
\label{sec:introduction}
\section{Motivations}
%2. What is the background, the context, in which the research took place? 
The concept of learning to count is an important educational/developmental milestone  which constitutes the most fundamental idea of mathematics. In Computer Vision\cite{umbaugh1997computer}, the counting problem is the estimation of number of objects in a still image or video frame. Learning to count visual objects is a new approach towards dealing with detecting object in the images and video, which has been recently proffered in the literature\cite{viola2005detecting, rabaud2006counting, kong2005counting, chan2008privacy, segui2015learning}. It arises in many real-world applications, including cell counting in microscopic images\cite{flaccavento2011learning}, monitoring crowds in surveillance systems\cite{rahmalan2006crowd, valera2005intelligent}, and performing wildlife census or counting the number of trees in an aerial image of a forest\cite{brandtberg1998automated, pollock1996automatic}\cite{NIPS2010_4043}. 

Artificial intelligence and computer vision share topics such as pattern recognition and machine learning\cite{michalski2013machine, mitchell1997machine} techniques. Consequently, computer vision is sometimes seen as a part of the artificial intelligence field or the computer science field in general. Recent machine learning methods applied for computer vision tasks, require large number of data for the learning process. To learn to count the object of interest in an image or video, various object features need to be designed, extracted or detected during the learning phase. The complexity of feature detection process in vision tasks, restrict their usage in large-scale computer vision applications thus demanding more efficient solutions to alleviate, expedite and improve this process. \\
\indent One recent and commonly used method to facilitate feature detection process is application of deep Convolutional Neural Networks(CNN)\cite{szegedy2015going, krizhevsky2012imagenet, lecun1995convolutional, sermanet2013overfeat, ji20133d, taylor2010convolutional}. One of the promises of deep CNN is replacing handcrafted features with efficient algorithms for unsupervised or semi-supervised feature learning and hierarchical feature extraction\cite{song2013hierarchical}. CNNs have been claimed and practically proven to achieve the most assuring performance in different vision benchmark problems\cite{ciresan2011flexible, szegedy2015going, ciresan2012multi}. 

\section{Objectives}
\section{Contributions of the Research}
This thesis contains the following contributions:
\begin{enumerate}
	\item It proposes the problem of object representation as an indirect learning problem casted as learning to count strategy. The devised algorithm is capable of counting the number of pedestrians in the image that does not depend on object detection or feature tracking. The model is also privacy-preserving in a sense that it can be implemented with hardware that does not produce a visual record of the individuals in the scene. 
	\item It provides a synthetically generated and automatically labeled dataset of pedestrians using unlabeled University of California San Diego(UCSD) pedestrian dataset used in \cite{mahadevan2010anomaly}, to train a counting deep convolutional neural network which is adequate for apprehending the underlying representations of the learned features. To this end, we describe a counting problem for MNIST dataset to demonstrate the capability of the internal representation of the network for classifying digits with no direct supervising while training. 
	\item The proposed model is able to count the number of people in the real and unseen dataset using the features learned by training the network on synthetic training set. To our knowledge, this is the first crowd counting system trained by synthetic data that successfully operates continuously on real data. 
	\item Along with the validation of our proposal in the following ways:
	\begin{itemize}
		\item First, we learn to count even hand-written digits using MNIST dataset. 
		\item Second, we validate the system quantitatively on a large synthetic dataset of pedestrian, containing 100,000 images with maximum 30 pedestrians in each image. 
		\item Last but not least, we count the number of pedestrians in the manually labeled dataset of 3375 images provided by[\citeauthor*{chan2013ground}, \citeyear{chan2013ground}]. 
	\end{itemize}
	
\end{enumerate}

\section{Organization}
%3. Why is this subject or issue important 


%4. Who are the key participants and/or ‘actors’ in the area under investigation? 

%5. Are there important trends or pivotal variables of which the reader needs to be made aware? 

%6. A clear and succinct statement of the aims and objectives that the dissertation is going to address. 

%7. Have you presented a clear and unambiguous exposition of your research aim, the objectives you will address to meet this aim and your research questions? 

%8. The reasons why this study was carried out 

%9. Was this study undertaken for example in order to test some aspect of  professional or business practice or theory or framework of analysis? 

%10. Was the research carried out to fulfil the demands of a business  organisation? 
%11. The way the Dissertation is to be organised 


