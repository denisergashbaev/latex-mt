\newpage
\chapter{State of the art review}
\label{sec:state}

Counting the number of an object of interest in an image can be approached from two different perspectives, either training an object detector, or training an object counter\cite{segui2015learning}. In the first case, we must provide the system with a large set of object examples, properly and almost in all cases manually labeled and localized in a way that represent most of the possible views and appearances of the object. The result is an sophisticated object classifier based on manually-crafted features\cite{viola2004robust, viola2005detecting}. In the latter case, we only need to provide the number of object instances for each image sample and the result is typically a regressor\cite{lempitsky2010learning}. Following this line of work, in \citeauthors{segui2015learning} in \cite{segui2015learning} proposed a novel approach to tackle 



%\cite{paragios2001mrf, cho1999neural, regazzoni1996distributed, davies1995crowd, kong2005counting, marana1998efficacy, viola2004robust}. 

