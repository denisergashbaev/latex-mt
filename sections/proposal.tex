\newpage
\chapter{Proposal}
\label{sec:proposal}
\noindent

This master thesis proposes an application of deep neural networks to a task of counting objects in the image. The novelty of our approach is that we hypothesize that features learned by training a counting deep CNN on synthetic dataset, are representative enough to count the number of object of interest on a real dataset. Hence, to verify our hypothesis and resolve the deficits of the state of the art, first we scale up  \citealt*{segui2015learning}' work along with a different architecture to observe if we attain promising results, and then we compare the performance of our model with the results \citeauthor{chan2008privacy} achieved on real dataset\cite{chan2008privacy}.  
To the best of our knowledge, the proposed work would be the first one in which a counting system built by trained with synthetic images is able to be incorporated in real-world similar counting problems.

\section{title}
 



%\begin{equation} \label{eq:curvilinear_energy}
%\begin{split}
%I \colon & \mathbb{R}^{D} \phantomarrow{AAAA} {}\mathbb{S}^1\times\mathbb{R}^2\\
%& \phantomarrow{AAA}{p} I(p) =(h,s,i)
%\end{split}
%end{equation}

%\begin{figure}[h!]
	%\centering
	%\begin{minipage}{0.45\textwidth}
		%\centering
		%{\includegraphics[width=0.8\textwidth]{images/image_in_question_modified.png}} % first figure itself
		%\caption{First step of the iteration in the segmentation. The initial cage in blue, the contour in white and the next cage in green with ten times the learning rate of step 1.} \label{fig:image_in_question}
	%\end{minipage}\hfill
	%\begin{minipage}{0.5\textwidth}
		%\centering
		%{\includegraphics[width=1.\textwidth]{images/gradient_plot_edit.png}}
		%\caption{Contribution of the points in a slice of image~\ref{fig:image_in_question} on the direction of the gradient with respect to $v_i$ of the x axis.}\label{fig:plot_in_question}
	%\end{minipage}
	%\centering
