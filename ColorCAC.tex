 \documentclass[11pt, oneside]{book}
\usepackage{setspace}
\newcommand{\HRule}{\rule{\linewidth}{0.5mm}}
%\usepackage{fancyhdr}
\usepackage[T1]{fontenc}
\usepackage[margin=1.1in]{geometry}
\usepackage[utf8]{inputenc}
\usepackage{epsfig}
\usepackage{subfigure}
\usepackage{calc}
\usepackage{dsfont}
\usepackage{amssymb}
\usepackage{amstext}
\usepackage{amsmath}
\usepackage{amsthm}
\usepackage{mathtools}
\usepackage{amsfonts}
\usepackage{multicol}
\usepackage{mathtools,xparse}
\usepackage[hidelinks]{hyperref}
\usepackage{booktabs}
\usepackage{enumitem}
\usepackage{calligra}

% % % % REFERENCES
%
\usepackage[numbers]{natbib}
\usepackage{amsmath}
%
% % % %

% % % %
%
\usepackage[labelfont=bf]{caption} %labelfont is to make the Figure X.X: in bold
\usepackage[titletoc,toc,page]{appendix}
%
% % % %

% % % % TableFootnote
%
\newcommand*{\savedfootnotes}{}
\newcommand*{\resetsavedfootnotes}{\global\let\savedfootnotes\empty}
\newcommand{\tablefootnote}[1]
{
	\footnotemark
	\xdef\savedfootnotes
	{\unexpanded\expandafter{\savedfootnotes}\noexpand\footnotetext{#1}}
}
\edef\endtable
{
	\aftergroup\noexpand\savedfootnotes
	\aftergroup\noexpand\resetsavedfootnotes
	\unexpanded\expandafter{\endtable}
}
%
% % % %

% % % % Norms
%
\DeclarePairedDelimiter{\abs}{\lvert}{\rvert}
\DeclarePairedDelimiter{\norm}{\lVert}{\rVert}
\NewDocumentCommand{\normL}{ s O{} m }{
	\IfBooleanTF{#1}{\norm*{#3}}{\norm[#2]{#3}}_{L_2(\Omega)}%
}
%
% % % %

\setcounter{secnumdepth}{3}
\setcounter{tocdepth}{3}

\newcommand\phantomarrow[2]{
	\setbox0=\hbox{$\displaystyle #1\to$}
	\hbox to \wd0{
		$#2\mapstochar
		\cleaders\hbox{$\mkern-1mu\relbar\mkern-3mu$}\hfill
		\mkern-7mu\rightarrow$}
	\,}

\graphicspath{{Images//},{sections//},{Images/qualitative_tests/}}
\input epsf

\subfigtopskip=0pt
\subfigcapskip=0pt
\subfigbottomskip=0pt


% % % Sections 
\usepackage{titlesec}
\titleformat{\chapter}[hang]{\Huge\bfseries}{\thechapter}{15pt}{\Huge\bfseries}
%\titlespacing{\chapter}{0pt}{-50pt}{20pt}


%Bigger elements in matrix
\makeatletter
\renewcommand*\env@matrix[1][\arraystretch]{%
	\edef\arraystretch{#1}%
	\hskip -\arraycolsep
	\let\@ifnextchar\new@ifnextchar
	\array{*\c@MaxMatrixCols c}}
\makeatother

% % DEFINITIONS, propositions enviornment
\newtheorem{definition}{Definition}
\newtheorem{proposition}{Proposition}
% % 


% ALGORITHMS
\usepackage{algorithm}
\usepackage{algorithmicx}
\usepackage{algpseudocode}
\usepackage{pifont}
%\algdisablelines



\newcommand{\old}{\texttt{algorithmic}}
\newcommand{\euk}{Euclid}
\newcommand\ASTART{\bigskip\noindent\begin{minipage}[b]{0.5\linewidth}}
	\newcommand\ACONTINUE{\end{minipage}\begin{minipage}[b]{0.5\linewidth}}
	\newcommand\AENDSKIP{\end{minipage}\bigskip}
\newcommand\AEND{\end{minipage}}
%\renewcommand{\algorithmicrequire}{\textbf{Input:}}
%\renewcommand{\algorithmicensure}{\textbf{Output:}}
\newcommand{\algorithmicbreak}{\textbf{break}}

\algdef{SE}[DOWHILE]{Do}{doWhile}{\algorithmicdo}[1]{\algorithmicwhile\ #1}%
% % % % % % % % % % % % % % % % % % % %

\begin{document}

%\title{Cage active contours: Extension to color spaces and
%		 application to image morphing }
%
%\keywords{The paper must have at least one keyword. The text must be set to 9-point font size and without the use of bold or italic font style. For more than one keyword, please use a comma as a separator. Keywords must be titlecased.}
%
%\abstract{This master thesis explores. experimental results confirm the superiority in results of
%	the algorithm compared to the previous existing energies. Furthermore,
%	we propose a pipeline for its application in morphing,}
%
%\onecolumn \maketitle \normalsize \vfill
\setcounter{page}{3}
%\pagestyle{fancy}
\begin{titlepage}
	
	\begin{figure}[t]
		\centering
		\includegraphics[scale=0.8]{images/upc_ub_urv}
	\end{figure}    
	
	
	\begin{center}
		
		{\large \uppercase{Universitat politècnica de Catalunya}} \medskip \\
		{\large \uppercase{Universitat de Barcelona}} \medskip \\
		{\large \uppercase{Universitat Rovira i Virgili}} \medskip \\
		\vspace*{1cm}
		\begin{center}
			\Large	\bf   Master in Artificial Intelligence
		\end{center}
		

		{\Large Master of Science Thesis} \vspace*{1.0cm} \\
		\HRule \\[0.4cm]
		{ \huge \bfseries MY MASTER THESIS TITLE }\\[0.4cm] % Title of your document

		\HRule \\[1.cm]
	\end{center}

	
	\thispagestyle{empty}
	\begin{center}
	\Large	\bf Hadi Keivan Ekbatani\\
	\end{center}
	\begin{center}
		{\large \uppercase{Facultat d'informàtica de Barcelona (FIB)}} \medskip \\
		{\large \uppercase{Facultat de Matemàtiques (UB)}} \medskip \\
		{\large \uppercase{Escola tècnica superior d'enginyeria (URV)}} \medskip \\
	\end{center}
	\vspace*{1.cm}
	\begin{center} 
		\begin{multicols}{2}	
		{\large Supervisor:}
		
		
		\medskip %\medskip\smallskip
		
		{\Large\bf Oriol Pujol Vila}
		
		\medskip %\medskip
		
		{Department of analysis\\
			and applied Mathematics,\\
			 Universitat de Barcelona (UB)}
		
		\medskip
		
		{\large Co-supervisor:}
		
		\medskip
				
		{\Large\bf Santiago Segui Mesquida}
			
		\medskip %\medskip

		{Department of analysis\\
			and applied Mathematics,\\
			Universitat de Barcelona (UB)}
		\end{multicols}
		\medskip\medskip\medskip\medskip\medskip
		
		% This has obviously to be changed
		
		February 01, 2016
	\end{center}
	
\end{titlepage}
\thispagestyle{empty}

\thispagestyle{empty}
\chapter*{Acknowledgments}
\thispagestyle{empty}
I would like to sincerely thank my supervisors Oriol Pujol and Santi Segui for their support, guidance and mentorship. I greatly appreciate their demanding and inquisitive scientific attitude, while keeping always a calm and positive mindset. They are, in my mind, an inspiring example of what university professors should be.

Furthermore I would also like to dedicate this master thesis to my beloved parents and sister for their unsparing supports. My gratitude knows no bounds.

 Another special acknowledgment must be made to my friends in the master: Denis, Jeroni, Philipp, Pablo, Lorenzo, Iosu, Ferran and many others for the endless studying hours, morning coffees after crunching the brutal assignments all night long, valuable and constructive discussions which allowed me carry out this program. 
 



\pagenumbering{arabic}
\setcounter{page}{0}
\clearpage

%Table of contents
\newpage
\pagenumbering{roman} % Roman numerals
\chapter*{Abstract}
\thispagestyle{empty}
The main purpose of this master thesis is to enhance the performance of Cage Active Contours (CAC) in the context of color image object segmentation as well as provide a theoretical framework on which to justify the potential applications of the segmentation produced. 

We present two new energy functions which expand on the previous energies in CAC, by using Mixture Gaussian Models to capture multiple components per region, and extend their applicability to two different color spaces: RGB and the Hue component of HSI/HSV. Furthermore, we provide a mathematical formalization of the components in CAC (\textit{cage} and \textit{contour}) in order to demonstrate and prove the good properties that allow for a segmentation to be used for shape description and for image morphing and warping. In addition, we provide a public implementation in Python for object segmentation of CAC with different energies along with tools for image morphing and warping.

In order to validate our improvements, both quantitative and qualitative tests are used on three different datasets. The results show that the energy in RGB produces a substantial improvement over the previously developed energies for CAC. On the other hand we observe that the energy in the Hue component performs poorly on the given datasets despite its robustness with respect to changes in illumination. In the theoretical part, we prove which initial conditions are needed for a segmentation to provide useful results in shape description. Finally, we provide examples of some possible applications of CAC in morphing supported by our theoretical conclusions.

\tableofcontents

\listoffigures


\newpage
\pagenumbering{arabic} % Normal numerals

\newpage\null\thispagestyle{empty}\newpage
%\part{title}
\chapter{Introduction}
\label{sec:introduction}
\section{Motivation}
 
% \todo{again, it's not a `milestone' as i believe. you need a different wording here}WHAT I MEAN HERE IS NoT WHAT WE DO AS PROJECT, IS THE FUNDAMENTAL CONCEPT OF LEARNING TO COUNT, LIKE WHEN KIDS LEARN TO COUNT NUMBERS
The fundamental concept of learning to count is an important educational/developmental milestone which constitutes the most fundamental idea of mathematics. In Computer Vision \cite{umbaugh1997computer}, the counting problem is the estimation of number of objects in a still image or video frame. Learning to count visual objects is a new approach towards dealing with detecting object in the images and video, which has been recently proffered in the literature \cite{viola2005detecting, rabaud2006counting, kong2005counting, chan2008privacy, segui2015learning}. It arises in many real-world applications, including cell counting in microscopic images \cite{flaccavento2011learning}, monitoring crowds in surveillance systems \cite{rahmalan2006crowd, valera2005intelligent}, and performing wildlife census or counting the number of trees in an aerial image of a forest \cite{brandtberg1998automated, pollock1996automatic,NIPS2010_4043}. 

Artificial intelligence and computer vision share topics such as pattern recognition and machine learning techniques \cite{michalski2013machine, mitchell1997machine}. Consequently, computer vision is sometimes seen as a part of the artificial intelligence field or the computer science field in general. Recent machine learning methods applied for computer vision tasks, require large number of data for the learning process. To learn to count the object of interest in an image or video, various object features need to be designed, extracted or detected during the learning phase. The complexity of feature detection process in vision tasks, restrict their usage in large-scale computer vision applications thus demanding more efficient solutions to alleviate, expedite and improve this process. 

\indent One recent and commonly used method to facilitate feature detection process is application of deep Convolutional Neural Networks(CNN) \cite{szegedy2015going, krizhevsky2012imagenet, lecun1995convolutional, sermanet2013overfeat, ji20133d, taylor2010convolutional}. One of the promises of deep CNN is replacing handcrafted features with efficient algorithms for unsupervised or semi-supervised feature learning and hierarchical feature extraction \cite{song2013hierarchical}. CNNs have been claimed and practically proven to achieve the most assuring performance in different vision benchmark problems concerning feature detection and classification \cite{ciresan2011flexible, szegedy2015going, ciresan2012multi}. 

\section{Objectives}
This work puts forward several objectives:
\begin{enumerate}

\item To abate extensive highly-specialized feature detection efforts by substituting deep CNN for learning features in problems concerning counting object of interest where no explicit information about what we are counting is given to the system, except for the object's multiplicity in the image.

\item To prevail over exhaustive data annotation required for fully supervised learning to count tasks (when using DL algorithms), by introducing synthetically created and fully automatically annotated datasets to train deep CNN with.   

\item To explore the behavior of proposed algorithms on synthetic datasets of different types and complexity, and compare the performance with state-of-the-art outcomes \cite{segui2015learning}.   

\item To analyze the performance of designed system (trained with synthetic dataset) on real-world crowd counting problems and compare the results with previously developed systems in state-of-the-art \cite{chan2008privacy}.   
 

\end{enumerate}
\section{Contributions}
Following the course of achieving fore-mentioned objectives, this dissertation contains the following contributions:

\begin{enumerate}
	\item It proposes the problem of object representation as an indirect learning problem casted as learning to count strategy. The devised algorithm is capable of counting the number of even hand-written digits in images. Moreover, the model is able to be applied for different but related tasks such as even-odd digit recognition, to demonstrate the capability of the learned features of the network for classifying digits with no direct supervising while training. 
	
	\item It provides a deep convolutional neural network for counting the number of pedestrians in a walkway that does not depend on object detection or feature tracking. The model is privacy-preserving in a sense that instead of tracking people, it learns the individual's features.
	
	\item It provides two synthetically created datasets for the proposed counting problems. First set of images contain hand-written digits for counting the number of even digits in the images. The second dataset is consist of pedestrians in a walkway, to train a counting deep convolutional neural network which is adequate for apprehending the underlying representations.
	 

	%\item It provides a synthetically generated and automatically labeled dataset of pedestrians using unlabeled University of California San Diego(UCSD) pedestrian dataset used in \cite{mahadevan2010anomaly}, to train a counting deep convolutional neural network which is adequate for apprehending the underlying representations of the learned features. To this end, we describe a counting problem for a synthetically created dataset of hand-written digits to demonstrate the capability of the internal representation of the network for classifying digits with no direct supervising while training. 
	\item The proposed model is able to count the number of people in the real and unseen dataset using the features learned by training the network on synthetic dataset. To our knowledge, this is the first crowd counting system trained by synthetic data that successfully operates on real data. 

	\item Validation of our proposal is done in the following steps:

	\begin{itemize}
		\item First, we learn to count even hand-written digits in images. Then, we evaluate the learned features of the proposed model on an even-odd digits recognition task. 
		\item Second, we validate a more complex model  quantitatively on a large synthetic dataset of pedestrians, containing maximum 29 people in each image. 
		\item Finally, we test our model's performance by counting the number of crowd in a real-world manually labeled dataset of people present in a walkway provided by~\citealt*{chan2013ground}. 
	\end{itemize}
	
\end{enumerate}

\section{Organization}

This report takes off with a review of Deep Learning(DL) (chapter~\ref{sec:dl}) as a branch of Artificial Intelligence (AI) which deep convolutional neural networks belong to, and moves on to introduce a deep CNN's basic architecture and components in details (chapter~\ref{sec:deepcnn}). 


Chapter ~\ref{sec:stateoftheart} reviews state-of-the-art of feature detection and learning to count problems. 

In Chapter~\ref{sec:proposal}, we describe our proposal for constructing a deep neural network to tackle feature detection issue learning to count problems. 

Chapter ~\ref{sec:implementation} introduces the applied platform to implement our methodology  along with the peculiarities of proposed data creation process and network modeling. 

Chapter ~\ref{sec:experiments} deals with the empirical experiments and analysis. Obtained results and comparisons with state-of-the-art solutions are expressed in this section.

Lastly, in Chapter~\ref{sec:conclusions}, we conclude the report with a short summary of the scope of work conducted and the new areas of research that this master thesis has opened.


\newpage
\chapter{State of the art review}
\label{sec:related_work}

Counting the number of an object of interest in an image can be approached from two different perspectives, either training an object detector, or training an object counter\cite{segui2015learning}. In the first case, we must provide the system with a large set of object examples, properly and almost in all cases manually labeled and localized in a way that represent most of the possible views and appearances of the object. The result is a sophisticated object classifier based on manually-crafted features\cite{viola2004robust, viola2005detecting}. In the latter case, we only need to provide the number of object instances for each image sample and the result is typically a regressor\cite{lempitsky2010learning}.  

Recently, with the success of CNNs in different vision tasks, object detection systems based on deep CNN have made groundbreaking advances on several object detection problems\cite{zhang2015improving, erhan2014scalable, girshick2014rich, he2015spatial, erhan2014scalable} which suggests the use of this technique to learn to count objects. Several advantages can be foreseen from this application, being the most important that of learning image features from samples instead of hand-crafting highly specialized image features that are dependent on the object class\cite{segui2015learning}. Moreover, CNN have shown their capacity of knowledge transfer for a number of tasks or the ability of simultaneously performing different tasks even when trained for only one \cite{zhou2014learning}. 

Following this line of work, \citefullauthor{segui2015learning} in \cite{segui2015learning} proposed a novel approach for counting objects' representations using deep object features. In their work, objects' features are learned by a deep counting convolutional neural network and are used to understand the underlying representation. To this end, they define a counting problem for even digits using \textit{MNIST} data and demonstrate that the internal representation of the network is able to classify digits in spite of the fact that no direct supervision was provided for them during training. Moreover, they present preliminary results about a deep network that is able to count the number of pedestrians in a scene\cite{segui2015learning}. Figure~\ref{fig:santimnist} illustrates their proposal at a glance in the case of representing hand-written digits:
\begin{figure}[h!]
	\centering
	{\includegraphics[width=0.6\textwidth]{images/santimnist}}
	\caption{Learning to count hand-written digits problem in which the features of a CNN that has been trained to count digits can be readily used for more specific classification problems and even to localize digits in an image\cite{segui2015learning}.}
	\label{fig:santimnist}
\end{figure}


In \cite{segui2015learning}, the main hypothesis is that the number of occurrence of objects in an image provide strong presentational information due to their possible discriminate appearance for a feature learning process to exploit. In order to verify this hypothesis, for both experiments, they considered networks of two or more convolutional layers (since CNNs instinctively handle feature learning\cite{lecun1989backpropagation}) consisting of convolutional filters, ReLU non-linearities, max-pooling layers and normalization layer,   followed by one or more fully connected layers (regarding the impressive classification performance on different benchmark problems\cite{krizhevsky2012imagenet, Karpathy_2014_CVPR, ciresan2011flexible})\cite{segui2015learning}. 







  


%\cite{paragios2001mrf, cho1999neural, regazzoni1996distributed, davies1995crowd, kong2005counting, marana1998efficacy, viola2004robust}. 



\newpage
\chapter{State of the art review}
\label{sec:related_work}

Counting the number of an object of interest in an image can be approached from two different perspectives, either training an object detector, or training an object counter\cite{segui2015learning}. In the first case, we must provide the system with a large set of object examples, properly and almost in all cases manually labeled and localized in a way that represent most of the possible views and appearances of the object. The result is a sophisticated object classifier based on manually-crafted features\cite{viola2004robust, viola2005detecting}. In the latter case, we only need to provide the number of object instances for each image sample and the result is typically a regressor\cite{lempitsky2010learning}.  

Recently, with the success of CNNs in different vision tasks, object detection systems based on deep CNN have made groundbreaking advances on several object detection problems\cite{zhang2015improving, erhan2014scalable, girshick2014rich, he2015spatial, erhan2014scalable} which suggests the use of this technique to learn to count objects. Several advantages can be foreseen from this application, being the most important that of learning image features from samples instead of hand-crafting highly specialized image features that are dependent on the object class\cite{segui2015learning}. Moreover, CNN have shown their capacity of knowledge transfer for a number of tasks or the ability of simultaneously performing different tasks even when trained for only one \cite{zhou2014learning}. 

Following this line of work, \citefullauthor{segui2015learning} in \cite{segui2015learning} proposed a novel approach for counting objects' representations using deep object features. In their work, objects' features are learned by a deep counting convolutional neural network and are used to understand the underlying representation. To this end, they define a counting problem for even digits using \textit{MNIST} data and demonstrate that the internal representation of the network is able to classify digits in spite of the fact that no direct supervision was provided for them during training. Moreover, they present preliminary results about a deep network that is able to count the number of pedestrians in a scene\cite{segui2015learning}. Figure~\ref{fig:santimnist} illustrates their proposal at a glance in the case of representing hand-written digits:
\begin{figure}[h!]
	\centering
	{\includegraphics[width=0.6\textwidth]{images/santimnist}}
	\caption{Learning to count hand-written digits problem in which the features of a CNN that has been trained to count digits can be readily used for more specific classification problems and even to localize digits in an image\cite{segui2015learning}.}
	\label{fig:santimnist}
\end{figure}


In \cite{segui2015learning}, the main hypothesis is that the number of occurrence of objects in an image provide strong presentational information due to their possible discriminate appearance for a feature learning process to exploit. In order to verify this hypothesis, for both experiments, they considered networks of two or more convolutional layers (since CNNs instinctively handle feature learning\cite{lecun1989backpropagation}) consisting of convolutional filters, ReLU non-linearities, max-pooling layers and normalization layer,   followed by one or more fully connected layers (regarding the impressive classification performance on different benchmark problems\cite{krizhevsky2012imagenet, Karpathy_2014_CVPR, ciresan2011flexible})\cite{segui2015learning}. 







  


%\cite{paragios2001mrf, cho1999neural, regazzoni1996distributed, davies1995crowd, kong2005counting, marana1998efficacy, viola2004robust}. 



\input{sections/proposed_methods}


\chapter{Implementation}
\label{sec:implementation}
\noindent
In this chapter, we provide a detailed implementation of our proposed methodology. We start by giving insight into the introduced algorithms for generating the synthetic datasets to train and test our models. Then we present the software platform we incorporated to shape and design our architectures. Finally, we demonstrate our networks' architecture and settings in detail. 

%\noindent In this chapter, we provide a detailed implementation of our proposed methodology. We start with presenting the software platform we incorporated to shape and design our models. Then we demonstrate our network's architecture in detail. Finally, we attempt to give insight into the datasets we used to train and test our model. 



\section{The Datasets}
\label{dataha}

Here we delve into the data processing part of this work by introducing three different datasets we created or applied for our empirical experiments. To that end, we provide a detailed explanation for each step of synthetic data generation process along with approaches and methods used to improve each dataset.

%Here we delve into the data processing part of this work by introducing three different datasets we generated or chose for our empirical experiments. To that end, we provide a detailed explanation of the approaches and methods used to generate and improve each dataset.

\subsection{Even-odd Digits Dataset}
\label{subsubsec:digit}

Our Even-odd handwritten dataset contains images of size $100\times100$. Each image is filled with from 0 up to 15 randomly selected digits from MNIST hand-written digits dataset. The process of creating images follows the steps in below:
\begin{enumerate}
\item MNIST Digits are resized to $18\times18$ pixels.
\item Up to 15 digits are randomly put in images of size $100\times100$ pixels.
\item The images are created with controlled overlapping by ensuring that two different numbers are 18 pixels away from each other, i.e. the distance between two digits centers is larger than 18 pixels.
\item Images are labeled with the number of even digits present in each image. 
\item The images are created uniformly, meaning that for example, the number of images containing 0 even digits is equal to the number of images containing 15 even digits.
\end{enumerate}
 This dataset has in total 1 million images: 800,000 images for training set and 200,000 as the test. Figure~\ref{fig:l2cmnist} illustrates some examples of even-odd digits dataset with different number of even digits in images.

\begin{figure}[H]
	\centering
	{\includegraphics[width=0.5\textwidth]{images/l2cimages}}
		\caption{An example of even-odd digits images. Form left to right, images contain 0 to 15 even digits.}
	\label{fig:l2cmnist}
\end{figure}

%Digits are resized to $18\times18$ pixels and randomly put in the image. The images are created with controlled overlapping by ensuring that two different numbers are 18 pixels away from each other, i.e. the distance between two digits centers is larger than 18 pixels. For the training process, images are labeled with the number of even digits present in each image. Figure~\ref{fig:l2cmnist} illustrates some examples of even-odd digits dataset with different number of even digits in images.

%Our Even-odd handwritten dataset contains images of size $100\times100$. Each image is filled with from 0 up to 15 randomly selected digits from MNIST hand-written digits dataset. Digits are resized to $18\times18$ pixels and randomly put in the image. The images are created with controlled overlapping by ensuring that two different numbers are 18 pixels away from each other, i.e. the distance between two digits centers is larger than 18 pixels. For the training process, images are labeled with the number of even digits present in each image. Figure~\ref{fig:l2cmnist} illustrates some examples of even-odd digits dataset with different number of even digits in images. 

%\begin{figure}[H]
%	\centering
%	{\includegraphics[width=0.9\textwidth]{images/l2cmnist}}
%		\caption{An example of even-odd digits images. Form left to right, images contain 0, 5, 10 and 15 even digits.}
%	\label{fig:l2cmnist}
%\end{figure}

%\indent This dataset has in total 1 million images: 800,000 images for training set and 200,000 as the test. Also, the dataset is uniformly generated, meaning that for instance, the number of images containing 0 even digits is \todo{`the numer' is always `is'} equal to the number of images containing 15 even digits.  

\subsection{Synthetic Pedestrians Dataset}
\label{subsec:synped}

Learning features using deep architectures requires a large amount of data. More importantly, for a fully supervised learning, this data should be annotated. Lack of data or its' high annotation cost prohibit the usage of deep learning methods for many problems including crowd counting. 

\indent However, in order to lower this cost, in our research, we decided to create a synthetic dataset of pedestrians in a walkway. To do that, we used UCSD unlabeled Anomaly detection dataset of pedestrians collected by \citeauthor{chan2008privacy} and used in \cite{chan2009analysis, mahadevan2010anomaly, li2014anomaly}. UCSD Anomaly detection dataset contains clips of groups of people walking towards and away from the camera, and consists of 34 training video samples and 36 testing video samples. Each video has 200 frames of each $238\times158$ pixels. Figure~\ref{fig:anomaly} depicts some images of UCSD Anomaly dataset.

\begin{figure}[H]
	\centering
	{\includegraphics[width=0.7\textwidth]{images/anomaly}}
	\caption{Sample images of UCSD Anomaly detection dataset.}
	\label{fig:anomaly}
\end{figure}


\indent In our study, we used all 70 training and testing video samples to generate the synthetic pedestrians dataset. To thoroughly demonstrate the generation process of our dataset, we divide this section into data generation and data improvement.

  
\subsubsection{Data Generation}

In our dataset, we constrained each image by having up to 29 pedestrians in the walkway. The process of generating the data includes the following steps:
\begin{enumerate}

\item \textbf{Background extraction:} Firstly, we simply subtract the background from each video frame. We extract two types of backgrounds: the median of all the backgrounds in each video (in total, 70 different backgrounds), and the median of all median backgrounds. An example of extracted backgrounds is shown below.

\begin{figure}[H]
	\centering
	{\includegraphics[width=0.4\textwidth]{images/background}}
	\caption{One background image extracted from UCSD dataset.}
	\label{fig:bgim}
\end{figure}


\item \textbf{Pedestrian extraction:} Subtracting each image from the mean background, we were able to label the connected regions (each individual in case of our images) of the subtraction using morphological labeling methods. Figure~\ref{fig:nobg} shows how images look like after background subtraction. 
\begin{figure}[H]
	\centering
	{\includegraphics[width=0.9\textwidth]{images/nobg}}
	\caption{Images after background subtraction.}
	\label{fig:nobg}
\end{figure}
 
Then, properties of labeled regions are measured and bound-boxed (see \cite{van2014scikit} for more detailed explanation). Boxes of people are center-based annotated. These labeled boxes shape our initial list of pedestrians with masks of the same size of each box. Figure~\ref{backback} in below demonstrates a selection of extracted pedestrians and pedestrians' masks.

\begin{figure*}[h!]
    \centering
    \begin{subfigure}[t]{0.5\textwidth}
        \centering
        {\includegraphics[width=0.5\textwidth]{images/peds}}
        \caption{Extracted pedestrians.}
    \end{subfigure}%
    ~ 
    \begin{subfigure}[t]{0.5\textwidth}
        \centering
        {\includegraphics[width=0.5\textwidth]{images/masks}}
        \caption{created pedestrians' masks.}
    \end{subfigure}
    \caption{Pedestrians and their corresponding masks.}
    \label{backback}
\end{figure*}


\item \textbf{Background generation:} In this step, we tried to make the backgrounds of images as realistic as possible by:
\begin{enumerate}
\item making a sparse combination of median backgrounds.
\item changing the global illumination of the images randomly.
\item adding some random Gaussian noise to the backgrounds.
\end{enumerate}
 As you may notice, in the figure~\ref{backback}, the generated background happened to be brighter and noisier than the extracted one.
 
 
%1)~making a sparse combination of median backgrounds, 2)~changing the global illumination of the images randomly, and 3)~adding some \textit{Gaussian noise} to the backgrounds. As you may notice, in the figure~\ref{backback}, the generated background happened to be brighter and noisier than the extracted one. 

\begin{figure*}[h!]
    \centering
    \begin{subfigure}[t]{0.4\textwidth}
        \centering
        {\includegraphics[width=0.8\textwidth]{images/background}}
        \caption{Raw extracted background image.}
    \end{subfigure}%
    ~ 
    \begin{subfigure}[t]{0.4\textwidth}
        \centering
        {\includegraphics[width=0.8\textwidth]{images/background3}}
        \caption{Generated background after a sparse combination, global illumination change and some noise.}
    \end{subfigure}
    \caption{A synthetically generated background image}
    \label{backback}
\end{figure*}

 Having backgrounds generated and pedestrians extracted and labeled, backgrounds are selected randomly. Then, for training and comparison purposes, images are masked with a filter of \textit{Region Of Interest}~(ROI). The mask and ROI is shown in below (figure~\ref{fig:roi}).

\begin{figure*}[h!]
    \centering
    \begin{subfigure}[t]{0.4\textwidth}
        \centering
        {\includegraphics[width=0.8\textwidth]{images/catwalk}}
        \caption{ROI mask.}
    \end{subfigure}%
    ~ 
    \begin{subfigure}[t]{0.4\textwidth}
        \centering
        {\includegraphics[width=0.8\textwidth]{images/roi}}
        \caption{ROI in the image.}
    \end{subfigure}
    \caption{Applying the mask of region of interest on the background image.}
    \label{fig:roi}
\end{figure*}

\item \textbf{Creating synthetic images:} Afterwards, pedestrians are added to the masked background in a way that the center of each person is placed inside white area of the mask. Finally images are normalized (between 0 and 255) and resized to $158\times158$ in order to be fed to convolution layers. A selection of created images are depicted in the figure underneath.

\begin{figure}[H]
	\centering
	{\includegraphics[width=0.8\textwidth]{images/myped}}
	\caption{created synthetic images for counting pedestrians problem.}
	\label{fig:myped}
\end{figure}

\end{enumerate}

\subsubsection{Data Improvement}
\label{dataimp}
Although we managed to successfully create synthetic images of people in the street, the generated images were still quite distinguishable from the real dataset. Thus, in order to make images as highly realistic as possible, we improved the dataset as explained underneath:
\begin{itemize}
\item \textbf{Non-pedestrian objects:} Amongst the extracted boxes of pedestrians, there were some non-pedestrian boxes with objects instead of pedestrians, and yet others with more than one person inside the box. Therefore, we manually removed these outliers. After this edition, we ended with 426 samples of people. A few examples of incorrectly collected or labeled objects are shown in below.

\begin{figure}[H]
	\centering
	{\includegraphics[width=0.6\textwidth]{images/nonped}}
	\caption{A selection of non-human or incorrectly labeled objects. As you may see, there are some images with "half a person" and some others with more than just one person in the image.}
	\label{fig:nonped}
\end{figure}
 
\item \textbf{Lack of pedestrians:} For the sake of generalization, we needed a decent variety of pedestrians in the images to train with. For this purpose, we created 2 versions of current pedestrians list, each darkened by the factor of 20\% from each other. 
\item \textbf{Halos around the pedestrians:} Due to lack of accuracy of the region measuring method, a fine layer of the background that pedestrians were extracted from, still remained around the pedestrians. In the created images, depending on where the person was placed, these thin layers appeared like a halo around the person. Figure~\ref{fig:haloim} illustrates some images with halos around the pedestrians. 
\begin{figure}[H]
	\centering
	{\includegraphics[width=0.8\textwidth]{images/halo}}
	\caption{A few synthetic images with halos around the pedestrians in the walkway.}
	\label{fig:haloim}
\end{figure}
 
To mitigate this issue, we tried two approaches:
\begin{enumerate}
\item \textbf{Morphological erosion:} Among morphological operations on image, we applied \textit{erosion} \cite{van2014scikit} to erode the pedestrians masks. In this way, the halos were ignored to some noticeable extent. 
\item \textbf{Poisson image editing:} Poisson image editing is a technique for seamlessly blending two images together fully automatically  \cite{perez2003poisson}. In addition to erosion, we tried Poisson image editing tool to remove the halos. However, due to our gray-scale and low-resolution images, this tool did not have a great impact on our images.      
\end{enumerate}
Afterwards, the images look similar to the ones shown in figure~\ref{fig:nohalo}.

\begin{figure}[H]
	\centering
	{\includegraphics[width=0.8\textwidth]{images/nohalo}}
	\caption{Some examples of images with no or less halo around the people in the images.}
	\label{fig:nohalo}
\end{figure}
 
\item \textbf{Image Perspective:} Since pedestrians of different sizes were put randomly in the images, we considered people's tallness perspective in the images. As you may observe, in the following figure~\ref{fig:nopers}, image perspective has not been applied in the images and hence, there are some pedestrians closer to the camera but really small and also some pedestrians quite tall at the end of walkway.

\begin{figure}[H]
	\centering
	{\includegraphics[width=0.8\textwidth]{images/nopers}}
	\caption{created images before considering image perspective.}
	\label{fig:nopers}
\end{figure}

Humans' height almost follows a Gaussian distribution \cite{subramanian2011height}. Therefore, with respect to \cite{subramanian2011height, garcia2007evolution}, we mapped individual's heights with the length of the walkway in the image, considering a Gaussian noise with mean $\mu = 0$ and $\sigma = 3.5$. The resulting images are demonstrated as follows.

\begin{figure}[H]
	\centering
	{\includegraphics[width=0.8\textwidth]{images/pers}}
	\caption{Synthetic images considering image perspective based on the real distribution of people's height. As you may see, after applying perspective, the images look more realistic. }
	\label{fig:pers}
\end{figure}


\end{itemize}

\noindent Thusly, we created a set of 1 million images, each of size $158\times158$ pixels with up to 29 pedestrians. We assigned 800,000 images for training set and 200,000 instances as the test set. We believe the created synthetic dataset of pedestrians is realistic enough to be able to represent a real-world crowd counting scenario. However, the images can be improved in different aspects and by using diverse image editing and manipulation techniques. 

\subsection{UCSD Crowd counting Dataset}
\label{subsec:datareal2}
To verify and validate our model, we used UCSD crowd counting dataset created by \citeauthor*{chan2008privacy} and used in \cite{chan2008privacy,chan2009bayesian,chan2012counting}. The dataset contains video of pedestrians on UCSD walkways, taken from a stationary camera. There are currently two  viewpoints available among which we used \textit{vidf} videos. All videos are 8-bit gray-scale, each video file has 200 video frames, with dimensions $238\times158$. In our experiment, the first 20 videos which are labeled with the number of pedestrians, were incorporated. The center point of each pedestrian defines its' location in the image. A selection of UCSD crowd counting images is shown in figure~\ref{fig:ucsdorg}.

\begin{figure}[H]
	\centering
	{\includegraphics[width=0.8\textwidth]{images/normalucsd}}
	\caption{Normal UCSD crowd counting dataset images.}
	\label{fig:ucsdorg}
\end{figure}

\indent Among the labeled images, we selected the ones in which the number of pedestrians does not exceed 29. Then, images were resized to  $158\times158$ pixels and normalized between 0 and 255. Hence, in total we have a dataset of 3375 real images which are masked with the same filter we used for synthetic pedestrians dataset (shown in figure~\ref{fig:roi}). At last, the images look like the following images.

\begin{figure}[H]
	\centering
	{\includegraphics[width=0.8\textwidth]{images/testucsd}}
	\caption{Real UCSD images after being masked and resized.}
	\label{fig:testucsd}
\end{figure}


\indent We will use this dataset to first, validate the performance of our model trained with synthetic data, on a real dataset, and then to do a comparison between the work done in \cite{chan2008privacy} and our approach.



\section{Caffe Deep Learning Platform}

Caffe is a clean and modifiable framework for state-of-the-art deep learning algorithms and a collection of reference models. The framework is a BSD-licensed C++ library with Python and MATLAB bindings for training and deployment of general-purpose convolutional neural networks and other deep models on commodity architectures \cite{jia2014caffe}. It powers on-going research projects and large-scale industrial applications in vision, speech and multimedia by CUDA \footnote{CUDA is a parallel computing platform and application programming interface (API) model created by NVIDIA \cite{cuda}, processing over 40 million images a day on a single K40 or Titan GPU \cite{jia2014caffe}.}  GPU computation.

Caffe is composed of two main components, models' architecture and design, and model optimization. In the rest of this section, we present the main components and parameters of both model implementation and optimization.
\subsection{Model Implementation and Design}
The main components of Caffe architecture are outlined below:
\begin{enumerate}

\item \textbf{Data storage:} Caffe stores and communicates data in 4-dimensional arrays called \textit{blobs}. Blobs provide a unified memory interface, holding batches of data, parameters, or parameter updates. Blobs conceal the computational overhead by synchronizing from the CPU host to the GPU device as needed. Figure~\ref{fig:blob} shows the blobs connected to a convolution layer implemented by Caffe.


\begin{figure}[H]
	\centering
	{\includegraphics[width=0.7\textwidth]{images/caffeconvlayer}}
	\caption{Input (data) and output (conv1) blobs of a convolution layer in a CNN implemented in Caffe.}
	\label{fig:blob}
\end{figure}


Caffe supports some data sources such as LevelDB or LMDB (Lightning Memory-Mapped Database), HDF5, MemoryData, ImageData, etc. However, large-scale data is stored in LevelDB databases since it reads the data directly from memory \cite{caffe}. 
\item \textbf{Layers:} A caffe layer takes blobs as input and yields one or more as output. In a network (as described in chapter~\ref{subsec:bp}), each layer plays two important roles: a forward pass that takes the inputs and produces the outputs, and a backward pass that takes the gradient with respect to the output, and computes the gradients with respect to the parameters and to the inputs, which are in turn back-propagated to earlier layers \cite{jia2014caffe}.

\indent Caffe supports an exhaustive set of layers, including the followings \cite{jia2014caffe}: 
\begin{enumerate}
	\item Convolution, pooling, fully connected, 
	\item Nonlinearities like rectified
	linear and logistic, local response normalization, element-wise operations, and 
	\item Losses like softmax and hinge
\end{enumerate}
\item \textbf{Networks and run mode:} Caffe ensures the correctness of the forward and backward passes for any directed acyclic graph of layers. A typical network begins with a data layer laying down at the bottom going up to the loss layer that computes tasks' objectives. The network is run on CPU or GPU independent of the model definition. 
\item \textbf{Training a network:} Training phase in Caffe is done by classical stochastic gradient descent algorithm. When training, images and labels pass through different layers lead into the final prediction into a classification layer that produces the loss and gradients which train the whole network. Figure~\ref{fig:caffe} illustrates a typical example of a Caffe network. 


\indent Finetuning, the adaptation of an existing model to new architectures or data, is a standard method in Caffe. Caffe  finetunes the old model weights for the new task and initializes new weights as needed. This capability is essential for tasks such as knowledge transfer \cite{donahue2013decaf}, object detection \cite{girshick2014rich}, and object retrieval \cite{guadarrama2014open}, \cite{jia2014caffe}.  
\end{enumerate}


\begin{figure}[H]
	\centering
	{\includegraphics[width=0.9\textwidth]{images/le2col}}
	\caption{From bottom-left to top-right, Lenet architecture for MNIST digit classification example of a Caffe network, where boxes represent layers and octagons represent data blobs produced by or fed into the layers \cite{jia2014caffe}.}
	\label{fig:caffe}
\end{figure}

\textbf{Justification}: We decided to use Caffe because, it addresses computation efficiency problems (as likely the fastest available implementation of deep learning frameworks at the time of performing this study, adheres to software engineering best practices, providing unit tests for correctness and experimental rigor and speed for deployment. It is also well-suited for research use, due to the well-implemented modularity of the code, and the clean separation of network definition (usually the novel part of deep learning research) from actual implementation\cite{jia2014caffe}. In addition, it provides a python wrapper which exposes the solver module for easy prototyping of new training procedures. 

\subsection{Model Optimization}

Our work in general is an optimization problem since we project the results as loss functions we try to minimize. For this reason, model optimization methods have a critical impact on the performance of the model. In Caffe, \textit{Solver} file orchestrates optimization by coordinating the network's forward inference and backward gradients to form parameter updates that attempt to improve the loss. Among Caffe solver methods, we use stochastic gradient descent (explained in chapter~\ref{subsec:sgd}). The use of SGD in deep convolutional neural network setting is motivated by the high cost of running back propagation over the full training set. SGD can overcome this cost and still lead to fast convergence. 

Although Caffe provides several optimization methods, in this section we briefly describe merely optimization methods and \textit{hyper-parameters}\footnote{Hyper-parameters govern the underlying system on a "higher level" than the primary parameters of interest. They are not model parameters that are learned during training phase and instead, they are set by the designer a priori.} incorporated in our proposed algorithms.


\subsubsection{Batch Size}

Since Caffe is trained using stochastic gradient descent, at each iteration, it computes the (stochastic) gradient of the parameters with respect to the training data and updates the parameters in the direction of the gradient. On the other hand, to compute the gradient w.r.t the input data, we need to evaluate all training samples at each iteration which is prohibitively time-consuming, specially when we are dealing with a great amount of data. 

\indent In order to overcome this issue, SGD approximates the exact gradient, in a stochastic manner, by sampling only a small portion of the training data at each iteration. This small portion is the \textit{batch}. In other words, the batch size defines the amount of training data we feed to the network at each iteration. The larger the batch size, the more accurate the gradient estimate at each iteration will be. 

\subsubsection{Learning Rate}
\label{learning rate}
Learning rate is a decreasing function of time. It's a common practice to decrease the base learning rate (base\_lr) as the optimization/learning process progresses. In Caffe, different learning policies exist among which we tried the followings:
\begin{itemize}
\item \textit{\textbf{Fixed}:} which always returns the base learning rate.

\item \textit{\textbf{inv}:} which returns $$base\_lr \times (1 + \gamma \times iteration) ^ {(-Power)}$$ where:\\\textit{ $\gamma$: the factor learning rate drops by.}\\\textit{power: another parameter to compute the learning rate.}

\item \textit{\textbf{step}:} that returns $$base\_lr \times \gamma ^ {floor(\frac{iteration}{step})}$$ where:\\ \textit{$\gamma$: the factor learning rate drops by.}\\\textit{step: the number of iteration at which the learning rate drops.} 
\item \textit{\textbf{multi-step}:} similar to step but it allows non uniform steps defined by step value.
\end{itemize}
Although there are numerous empirical studies and rules of thumb to treat learning rate \cite{senior2013empirical,yu1995dynamic,minai1990acceleration}, basic learning rate and learning policy are highly problem-dependent.  

\subsubsection{Weight Decay}


As a part of Back Propagation and a subset of regularization methods, \textit{weight decay} adds a penalty term to the error function by multiplying weights to a factor slightly less than 1 after each update. 

 It has been observed in numerical simulations that a weight decay can improve generalization in a feed-forward neural network. It is proven that weight decay has two effects in a linear network. Firstly, it suppresses any irrelevant components of the weight vector by choosing the smallest vector that solves the learning problem. Secondly, if the size is chosen correctly, a weight decay can suppress some of the effects of static noise on the targets, which improves generalization significantly \cite{moody1995simple}. 

\subsubsection{Momentum}
%\todo{doesn't it belong to state-of-the art / implementation??cant put it in implementation cuz its built-in caffe and I just set the value. Hence, I thought they are better off here since they belong also to model optimization?}
One of the potential problems with stochastic gradient descent is having oscillations in the gradient, since not all examples are used for each calculation of the derivatives. This can cause slow convergence of the network. One strategy to mitigate this problem is the use of \textit{Momentum}. The momentum method introduced by \citeauthor{polyak1964some}, \citeyear{polyak1964some}, is a first-order optimization method for accelerating gradient descent that accumulates a velocity vector in directions of persistent reduction in the objective across iterations. Given an objective function $f(\theta)$ to be minimized, momentum is given by:
\begin{equation}
	\label{eq:t}
	\begin{aligned}
		\nu_{t+1} = \mu\nu_t - \alpha\nabla 
	\end{aligned}
\end{equation}
\begin{equation}
	\label{eq:t}
	\begin{gathered}
	\theta_{t+1} = \theta_t + \nu_{t + 1}
	\end{gathered}
\end{equation}
where $\alpha > 0$ is the learning rate, $\mu \in [0,1]$ is the momentum coefficient, and $\nabla f(\theta_t)$ is the gradient at $\theta_t$ \cite{sutskever2013importance}. 

For example, if the objective has a form of a long shallow ravine leading to the optimum and steep walls on the sides, standard SGD will tend to oscillate across the narrow ravine since the negative gradient will point down one of the steep sides rather than along the ravine towards the optimum \cite{sgd}. The objectives of deep architectures have this form near local optima and thus standard SGD can lead to very slow convergence particularly after the initial steep gains. 

Momentum, by taking the running average of the derivatives, by incorporating the previous update in the update for the current iteration, is one method for pushing the objective more quickly along the shallow ravine \cite{sgd}. 

\subsubsection{Number of Iterations}

In learning process, common convergence criteria are: a maximum number of iteration; a desired value for the cost function is reached; or training until the cost function shows no improvement in a number of iterations. In our implementation, we use the maximum number of iteration as our systems' convergence policy. 
 
The number of iterations plays an important role in the training process. iteration number is in an inverse correlation with the number of instances and also the batch size. In any network, batch size and iteration number compensate for one another. For instance, in case of lack of memory, one option would be to decrease the batch size and increase the number of iterations accordingly.


\section{The Architecture}
\label{imparch}

Having Caffe platform introduced, we propose two CNN-based deep architectures for the analysis we did regarding two learning to count problems, counting the number of even-digits and the number of pedestrians in an image.  
In learning object features for vision tasks and by the use of DCNN, the depth of network plays a crucial role. The deeper the model, the better it learns. However, issues like overfitting\footnote{Overfitting occurs when a statistical model describes random error or noise instead of the underlying relationship, and vice versa for underfitting. } and underfitting should not be left neglected.   
Therefore, in this section, networks' settings and architectures for even-digits and crowd counting problems will be described separately. 

\subsection{Even-digit Counting}
\label{subsubsec:digitarch}

For learning to count even digits problem, since we used MNIST dataset to generate our images, we considered the architecture proposed by \citeauthor{lecun1995comparison} for classic MNIST hand-written digit recognition problem \cite{lecun1995comparison} (shown in figure~\ref{fig:caffe}), as the base of our design. In addition, we took into account the architecture used in state-of-the-art \cite{segui2015learning}. Figure~\ref{santil2cfull} demonstrates this architecture. From there, we modified the architecture to optimize the performance of the network. 
 
\begin{figure}[H]
  \centering
   {\includegraphics[width=1.\textwidth]{images/santil2cfull}}
  %\end{center}
	\caption{From top-left to bottom-right, the network proposed in \cite{segui2015learning} for Even digits recognition task.}
	\label{santil2cfull}
\end{figure}

\noindent In a similar experiment with maximum 5 digits in images, \citeauthor{segui2015learning} in \cite{segui2015learning} defined a four layers CNN with two convolutional layers for classifying even digits present in images. Each convolutional layer consists of several elements: a set of convolutional filters, ReLU non-linearities, max pooling layers and normalization layers. The first convolutional layer outputs 10 filters of dimensions $15\times15$, and the second convolution layer has 10 filters, each of size $3\times3$. Each convolution layer is followed by a max-pooling layer with a $2\times2$ kernel. The output of last pooling layer is fed to two fully-connected layers with respectively 32 and 6 number of outputs. They also used Softmax loss layer on top of their architecture to compute the multinomial logistic loss layer\footnote{This function displays a similar convergence rate to the hinge loss function, and since it is continuous, gradient descent methods can be utilized. Also, functions which correctly classify points with high confidence are penalized less. This structure is highly sensitive to outliers in the data.}. Table~\ref{santil2c} presents the main components of the network designed by \cite{segui2015learning}. 

\begin{table}[H]
	\centering
	\begin{tabular}{ |p{2cm}|p{2cm}| }
	\hline 
	\multicolumn{2}{|c|}{\textbf{Network parameters}} \\
	\hline
	\hline
	\textbf{Layers} & \textbf{setting }\\
	\hline
	Conv1 & $10\times15\times15$\\
	\hline
	Pool1    & $max(2\times2)$ \\
	\hline
	Conv2 & $10\times3\times3$\\
	\hline
	Pool2 &    $max(2\times2)$ \\
	\hline
	FC1 & 32 outputs \\
	\hline
	FC2 & 6 outputs \\
	\hline
	\end{tabular}
		\caption{The DCNN proposed by \cite{segui2015learning} for learning the number of even digits present in the images.}
		\label{santil2c}
\end{table}
 

\noindent However, in our implementation, due to higher complexity, we decided to apply a deeper CNN with more parameters. In our network, the data layer fetches the images and labels from the disk, passes it through the first convolutional layer with 20 filters, each of size $15\times15$ followed by a ReLU non-linearity and LRN normalization layer. Then the output is max-pooled by the  kernel size of $2\times2$. This process repeats again but this time with the second convolutional layer having 50 filters of size $3\times3$. In all convolution and pooling layers, the \textit{stride} = 1 and \textit{padding}\footnote{Zero padding is a simple concept; it simply refers to adding zeros to end of an image to increase its length.} = 1 are considered (see section~\ref{convlayer} for more explanation regarding padding and stride). 

\noindent the output of the second pooling layer is fed to two fully connected (inner product) layers with respectively 64 and 1 number of outputs (since the problem is approached as a regression task). Also the first fully connected layer is followed by ReLU non-linearity. 


However, a well-designed network cannot solely guarantee an optimal performance for the model. The responsibilities of learning are divided between the network for yielding loss and gradients, and the optimization methods (solver) and parameters for overseeing the optimization and generating parameter updates. Among Caffe solvers, As described in section~\ref{subsec:sgd}, we use Stochastic Gradient Descent optimization method. Apart from solver method, the solver parameters (network's hyper-parameters) need to be set attentively in order to optimize the model performance. Therefore, to optimize the model performance, the networks' hyper-parameters are set as below while table~\ref{hypers} provides a summary of the hyper-parameters' settings. 
\begin{itemize}

\item \textbf{Batch size:} Due to the non-complex and low-resolution dataset we are training on, we were able to use batches of size 256 for our training and testing phases.

\item \textbf{Learning rate:} After trying different initial values in range of $(10^{-6}, 1)$, we set the basic learning rate to $\alpha = 0.0001$. However, for our experiment we chose \textit{multi-step} learning policy in which, after each \textit{stepsize}=40000 iterations, the learning rate drops by the rate of Gamma $\gamma = 0.1$. This initialization is based on rules of thumb used in \cite{krizhevsky2012imagenet}.

\item \textbf{Momentum:} We use momentum $\mu = 0.9$. Because, momentum setting $\mu$ effectively multiplies the size of our updates by a factor of $\frac{1}{1-\mu}$. Hence, changes in momentum and learning rate ought to be accompanied with an inverse correlation. When momentum $\mu = 0.9$, we have an effective update size of 10 since we also drop the learning rate by the factor of $\gamma= 0.1$.

%
%\item \textbf{Momentum:} We use momentum $\mu = 0.9$. This selection also is based on trial and error. As momentum value $\mu$ effectively multiplies the size of our updates by a factor of $\frac{1}{1-\mu}$. Hence, changes in momentum and learning rate ought to be accompanied with an inverse correlation. When momentum $\mu = 0.9$, we have an effective update size of 10 since we also drop the learning rate by the factor of Gamma $\gamma= 0.1$.

\item \textbf{Weight decay:} Weight decay as a penalty term to the error function, has a constant value of 0.0005. This decay constant is multiplied to the sum of squared weights.

\item \textbf{Iterations:} Given the time and hardware we had, we managed to let the system train for 1,600,000 iterations.

\end{itemize}

\begin{table}[H]
	\centering
	\begin{tabular}{ |p{3.8cm}|p{1.7cm}| }
	\hline 
	\multicolumn{2}{|c|}{\textbf{Network hyper-parameters}} \\
	\hline
	\hline
	\textbf{Hyper-parameters} & \textbf{setting }\\
	\hline
	Learning rate & 0.0001\\
	\hline
	Learning policy    & \textit{Multi\_step} \\
	\hline
	Momentum & $\mu = 0.9$\\
	\hline
	Weight decay & 0.0005 \\
	\hline
	Batch size & 256 \\
	\hline
	Iterations & 1600000 \\
	\hline
	\end{tabular}
		\caption{Proposed settings for network's hyper-parameters for counting number of even digits task.}
		\label{hypers}
\end{table}
 

\noindent We should also mention that at the top layer of the network, we used Euclidean Loss layer to compute the euclidean distance between the predictions and the ground truth.  Figure~\ref{fig:l2cNet} shows a scheme of the architecture while table~\ref{ourltc} summarizes the design. 
		\begin{table}[H]
			\centering
			\begin{tabular}{ |p{2cm}|p{2cm}| }
			\hline 
			\multicolumn{2}{|c|}{\textbf{Network parameters}} \\
			\hline
			\hline
			\textbf{Layers} & \textbf{setting }\\
			\hline
			Conv1 & $20\times15\times15$\\
			\hline
			Pool1    & $max(2\times2)$ \\
			\hline
			Conv2 & $50\times5\times5$\\
			\hline
			Pool2    & $max(2\times2)$ \\
			\hline
			Conv3 & $50\times3\times3$\\
			\hline
			IP1 & 128 outputs \\
			\hline
			IP1 & 64 outputs \\
			\hline
			IP2 & 1 outputs \\
			\hline
			\end{tabular}
				\caption{Proposed architecture's settings.}
				\label{ourltc}
		\end{table}
		 
%    \end{minipage} 
%        %\caption{Global caption}
%\end{table}



\begin{figure}[H]
  \centering
   {\includegraphics[width=0.9\textwidth]{images/ohoh}}
  %\end{center}
	\caption{From bottom-left to top-right, the proposed network architecture for Even digits counting task}
	\label{fig:l2cNet}
\end{figure}

%\todo{this is a useless statement, unless you go into detail what heuristics you've used. justification needed for your choise. so beef it up here between these two sentences}
The architecture has been designed intuitively and based on a heuristic approach where we tried different values for each hyper-parameter at a time and selected the values that provided us with an improvement in the performance of the network. Hence, one may be able to improve the performance by implementing a different architecture or with another settings for the hyper-parameters of the network. 

\subsection{Crowd Counting}
\label{subsec:ucsdarch}

For the task of counting pedestrians, due to the more complex images than the previous task, we designed a deeper architecture. Moreover, we applied the same settings of hyper-parameters to a different architecture. However, as a reminder, table~\ref{hypar2} summarizes the network's settings for the hyper-parameters.

\begin{table}[H]
	\centering
	\begin{tabular}{ |p{3.8cm}|p{1.7cm}| }
	\hline 
	\multicolumn{2}{|c|}{\textbf{Network hyper-parameters}} \\
	\hline
	\hline
	\textbf{Hyper-parameters} & \textbf{setting }\\
	\hline
	Learning rate & 0.0001\\
	\hline
	Learning policy    & \textit{Multi\_step} \\
	\hline
	Momentum & $\mu = 0.9$\\
	\hline
	Weight decay & 0.0005 \\
	\hline
	Batch size & 256 \\
	\hline
	Iterations & 1600000 \\
	\hline
	\end{tabular}
		\caption{Proposed settings for network's hyper-parameters for counting pedestrians.}
		\label{hypar2}
\end{table} 

As a baseline, we considered state-of-the-art architecture designed by \citet{segui2015learning} for counting the number of pedestrians in the images. As shown in figure~\ref{santiucsdnet}, in their work, they used a five layer architecture CNN with two convolutional layers followed by three fully connected layers. Convolutional layers contain their main elements (ReLU and LRN layers), and each outputs 8 filters with kernel sizes $9\times9$ and $5\times5$ respectively. 

\begin{figure}[H]
  \centering
   {\includegraphics[width=1.00\textwidth]{images/santiucsd}}
  %\end{center}
	\caption{From top-left to bottom-right, the DCNN proposed by \cite{segui2015learning} for learning to count the number of pedestrians in a walkway.}
	\label{santiucsdnet}
\end{figure}

They applied only one max-pooling layer and after the first convolutional layer in their design. Finally, three fully connected layers with number of outputs 128, 128 and 25 classify the output of the network into 25 classes (maximum pedestrians in the images) of the number of people in the image. Similar to the previous task, a softmax loss layer projects the performance of the model.   
Table~\ref{santiarch} describes a summary of their network.

\begin{table}[H]
	\centering
	\begin{tabular}{ |p{2cm}|p{2cm}| }
	\hline 
	\multicolumn{2}{|c|}{\textbf{Network parameters}} \\
	\hline
	\hline
	\textbf{Layers} & \textbf{setting }\\
	\hline
	Conv1 & $8\times9\times9$\\
	\hline
	Pool1    & $max(2\times2)$ \\
	\hline
	Conv2 & $8\times5\times5$\\
	\hline
	FC1 & 128 outputs \\
	\hline
	FC2 & 128 outputs \\
	\hline
	FC3 & 25 outputs \\
	\hline
	\end{tabular}
		\caption{The deep CNN proposed by \citealt{segui2015learning} for learning the number of pedestrians in a walkway.}
		\label{santiarch}
\end{table}


\noindent However, since up to 29 pedestrians are present in our images with high amount of overlapping, in our design we added one convolutional layer to the base architecture. We also increased the number of parameters in the network. In our architecture, the data blobs pass through 4 convolutional layers. The first convolutional layer has 10 filters, each with $15\times15$ kernel. The second convolution layer contains 10 filters of dimensions $11\times11$. Respectively for the third and fourth convolution layers, kernels of sizes $9\times9$ and $5\times5$  provide 0 filters for each layer. Similar to the previous model, each convolutional layer is followed by ReLU non-linearity layer and LRN normalization layer. Also the stride and padding values for all the convolutional layers are respectively equal to 1 and 0. Table~\ref{ourpednet} shows the main structure of this architecture. 

\begin{table}[H]
	\centering
	\begin{tabular}{ |p{2cm}|p{2cm}| }
	\hline 
	\multicolumn{2}{|c|}{\textbf{Network parameters}} \\
	\hline
	\hline
	\textbf{Layers} & \textbf{setting }\\
	\hline
	Conv1 & $10\times15\times15$\\
	\hline
	Pool1 & $max(2\times2)$ \\
	\hline
	Conv2 & $10\times11\times11$\\
	\hline
	Pool2 & $max(2\times2)$ \\
	\hline
	Conv3 & $20\times9\times9$\\
	\hline
	Conv4 & $20\times5\times5$\\
	\hline
	IP1   & 128 outputs \\
	\hline
	IP2   & 64 outputs \\
	\hline
	IP3   & 1 outputs \\
	\hline
	\end{tabular}
		\caption{Proposed architecture's settings.}
		\label{ourpednet}
\end{table}
%\todo{ok, i really want a figure showing the architecture of the network you've designed. it should show the layers, kernels, ... a zoom-in, no full-scale representation needed, just a cutout}

\indent In order to not lose information, we used merely two pooling layers for the first two convolutional layers. Each pooling layer has a kernel size of $2\times2$ with $stride = 1$  and $padding = 0$. Figure~\ref{fig:ucsdnet} illustrates the proposed architecture. 

\begin{figure}[H]
  \centering
   {\includegraphics[width=0.65\textwidth]{images/uscdmine}}
  %\end{center}
	\caption{Proposed network architecture for learning the number of pedestrians in the image.}
	\label{fig:ucsdnet}
\end{figure}

As you may see in figure~\ref{fig:ucsdnet} above, there are three fully connected layers to regress the number of pedestrians in images. The first fully connected layers has 128 output units, and the second layer outputs 64 units. The last layer however, with solely one output, passes the models' prediction of the number of pedestrians to the Euclidean loss layer to calculate the sum of squares of differences of its two inputs, the true labels and predictions. 
%In figure~\ref{fig:ucsdnet}, an illustration of this architecture has been presented.

To the best of our knowledge, the designed architectures outperform other architectures we tried previously (a different range of architectures starting from 1 up to 5 convolutional layers with different settings), while also fastening the training phase . However, apart from the basic knowledge about network architectures, hyper-parameters initialization and some rules of thumb of successful experiences in similar works, the rest of design has been done intuitively. Therefore, similar to the first architecture (problem of counting even digits), there would be other ways to improve the architecture and performance of the model.




\newpage
\chapter{Experiments and results}
\label{sec:experiments}

Prior to jumping to the final results obtained with our algorithm, this
chapter will attempt to give insight into the experiments we designed to verify our hypothesis. We examine our approaches on two different but related scenarios. Thus, we divide this section into two subsections where each presents, one experiment to test our methods, a comparison on the obtained results and conclusions about their performances and their properties.  

\section{Learning to count Even-odd hand-written digits}

We attempt to explore the features learned when training a deep CNN, in order to understand their underlying representations. To this end, we designed a synthetic problem of counting even digits in images. This experiment clearly illustrates the basic idea behind this work, where we hypothesize that the features learned during this task are:
\begin{enumerate}
\item Efficient descriptors of digits to enable the model to count them.
\item Sufficiently representative of the digits to be applied for digits recognition tasks. 
\end{enumerate}

In this experiment, we feed images of digits into a deep convolutional neural network. Images are labeled with the number of even digits in each image. We apply a regression strategy to tackle this problem. Therefore, the output of the network is a single number determining the model's prediction for the amount of even digits present in the images. 

\indent Having the experiment introduced briefly, we go over different steps of this analysis. After a succinct summary of the dataset, we explain training and testing the model. Then we provide the attained results along with a comparison to similar work and finally the conclusion we draw from this experiment. 

\subsection{Dataset} 
As it was meticulously described in (section ~\ref{subsubsec:digit}), we synthetically generated \textit{Even-odd digits dataset} to use for this task. the dataset holds the following properties:
\begin{itemize}
\item Images of MNIST hand-written digits where each image contains up to 15 digits. Also the images are made in gray-scale.
\item Each image has a dimension of 100*100. And each digit in the image is 18*18 pixels.  
\item A minimum distance of 18 pixels is considered between each two digits (from center to center) in an image to prevent overlapping.
\item The dataset is generated uniformly. It means that there are equal number of images for each amount of digits in the image. For instance, the number of images with 5 even digits are the same as number of images containing 10 even digits.
\item The dataset of 1,000,000 samples is divided into a training set of 800,000 and a test set of 200,000. 
\end{itemize}

Since we use Caffe platform to do our experiments, we convert the images into LMDB format. LMDB uses memory-mapped files, so it has the read performance of a pure in-memory database while still offering the persistence of standard disk-based databases, and is only limited to the size of the virtual address space, (it is not limited to the size of physical RAM). Therefore, we created two LMDB files for training and testing sets to be fed to the network.  

\subsection{Learning process}

As discussed in chapter~\ref{subsubsec:digitarch}, for learning to count the number of even digits, we designed a deep CNN. Table~\ref{fig:caffe} shows a summary of the network's specification and architecture.

\begin{figure}[H]
\centering
\begin{tabular}{ |p{2cm}|p{2.5cm}| }
\hline 
\multicolumn{2}{|c|}{\textbf{Network parameters}} \\
\hline
\hline
\textbf{Layers} & \textbf{setting }\\
\hline
Conv1 & 20*15*15\\
\hline
ReLU1 & max(x,0)  \\
\hline
LRN1 & $\alpha$=0.0001, $\beta$=0.75\\
\hline
Pool1    & max(2*2) \\
\hline
Conv2 & 50*3*3\\
\hline
ReLU2 & max(x,0)  \\
\hline
LRN2 & $\alpha$=0.0001, $\beta$=0.75\\
\hline
Pool2    & max(2*2) \\
\hline
IP1 & 64 outputs \\
\hline
ReLU3 & max(x,0)  \\
\hline
IP2 & 1 outputs \\
\hline
\end{tabular}
\end{figure}

However, a well-designed network solely cannot guarantee an optimal performance for the model. The responsibilities of learning are divided between the Net for yielding loss and gradients, and the optimization methods and parameters for overseeing the optimization and generating parameter updates. 

\indent In Caffe, \textit{Solver} file  orchestrates model optimization by coordinating the network’s forward inference and backward gradients to form parameter updates that attempt to improve the loss. As described in section~\ref{subsec:sgd},we use Stochastic Gradient Descent optimization method. In addition to the optimization method, the optimization parameters we set to obtain the optimum performance from the model, are shortly defined and 












Applying this network, we trained with 800,000 samples and tested over 200,000 images. 
Moreover, a schematic of the model with the output of each layer is depicted in figure~\ref{fig:digitnet}.










\begin{figure}[H]
	\centering
	{\includegraphics[width=0.9\textwidth]{images/digitnet}}
	\caption{An MNIST digit classification example of a Caffe network, where blue boxes represent layers and yellow octagons represent data blobs produced by or fed into the layers\cite{jia2014caffe}.}
	\label{fig:digitnet}
\end{figure}

\subsection{Results}
\subsection{Conclusion}

\section{Counting pedestrians in a walkway}

\newpage

\chapter{ Shape categorization and application to morphing}
\label{sec:applications}

The original use of the cage in the algorithm is to parametrize the image and to deform the
contour during the evolution to minimize the energy. We want to emphasize the advantages of having the cage parametrization and its properties. This chapter provides some first results of the future research for Cage Active Contours.


\section{Properties of the cage parametrization}
\label{sec:properties_parametrization}

As a motivation for this line of research, we propose applications of the cage parametrization in image morphing and warping and as shape descriptors. 

To formalize these applications we need a way to compare similar contour shapes. If we fix an initial contour and a cage configuration, we will get that for every new cage, a deformation of the initial contour by equation~\eqref{eq:deform_points} defines a deformed contour shape. Through observation, we see that similar cages, provide similar contour images in certain initial conditions. We want to find a criteria which allows us to link an ordered configuration of points (i.e. a cage) with contour shapes so that we may use existing tools from polygon similarity\cite{Bykat:1979:PS}  and from point configuration similarity, like Procrustes analysis distance, to determine shape similarity between different contours.

More formally, we present the following definitions which will lead up to the proposition and its proof.

\begin{definition}[Contour Family]\label{def:contour_family}
	Given an initial contour $C$ and an initial cage $V=(v_1,v_2,\dots,v_N)$, the family of contours $\mathcal{F}_C^V$ is the set of all the possible contours that can be produced with all cages of $N$ points by a deformation through \eqref{eq:deform_points} and it is expressed as:
	$$\mathcal{F}_C^V=\{ C^W\vert W \in (\mathbb{R}^2)^N \} $$
	where for any cage $W \in (\mathbb{R}^2)^N$ 
	$$C^W=\{q\in \mathbb{R}^2\vert q=\sum\limits_i^N\varphi_i^V(p)w_i ,\  \forall p \in C, W=(w_1, w_2, \dots, w_N)\}$$  
	and $\varphi_i^V(p)$ are the mean value coordinates of p with respect to cage V.
\end{definition}

\begin{definition}[Similarity]
	We define a similarity on the plane as an affine transformation $f:\mathbb{R}^2 \to \mathbb{R}^2$ composed of rotations, translations and uniform changes in scale. 
\end{definition}
\begin{definition}[Contour similarity]
	Two contours are similar if there exists a similarity which maps one to the other.
\end{definition}
\begin{definition}[Cage similarity]\label{def:cage_similarity}
	Two cages $U=(u_1, u_2, \dots, u_N)$ and $W=(w_1,w_2, \dots, w_N)$ are similar if there exists a similarity function such that $f(u_i)=w_i$ for each $i \in \{1,2,\dots, N\}$.
\end{definition}
\begin{definition}[Shifted cage]
	A shifted cage of another cage $W=(w_1,w_2, \dots, w_N)$ is a permutation conserving the order of $W$. There are N shifts (as many as number of points).
	$$W=W^0=(w_1,w_2, \dots, w_N)$$
	$$W^1=(w_2,w_3, \dots, w_N, w_1)$$
	$$\dots$$
	$$W^k=(w_{k+1},w_{k+2}, \dots, w_{k})$$
	$$\dots$$
	$$W^{N-1}=(w_N,w_1, \dots, w_{N-2}, w_{N-1})$$
\end{definition} 

In definition~\ref{def:contour_family} we have defined the contour Family of an initial configuration of a contour $C$ and a cage $V$. However, there are certain properties that we would like to impose on this Family. Namely, we are interested in those Families where similar cages or similar shifted cages define the same contour. 

First we need a definition:
\begin{definition}\label{def:initial_config}
	A \textbf{regular initial cage-contour configuration with ratio r} is a set ($V$, $C$, $r$) consisting of an initial cage $V=(v_1,v_2, \dots, v_N)$ that defines an N-sided regular polygon and an initial contour $C$ that is a circumference concentric to the polygon such that the ratio of the radius of $C$ and the radius of the polygon is r:1. For simplicity we say the ratio is r.
\end{definition}

Now that we have these concepts formally and well defined, we are able to prove the desired property of the family:

\begin{proposition}\label{prop:prop1}
Given a regular initial cage-contour configuration ($V$,$C$,$r$), then for every contour $C^W$ and  $C^U$ in the contour family $F_V^C$, $C^W$ and $C^U$ are similar if
\begin{enumerate}
	\item $W$ and $U$ are similar cages, \\
	or
	\vspace{-0.25cm}
	\item $U$ is a \textit{shifted} cage of a similar cage of $W$.
\end{enumerate} 
\end{proposition}

\begin{proof}
	The first point is trivial. We want to see if there exists a similarity function g that sends $C^W$ to $C^U$. So for every point of $q^W\in C^W$ there has to exist a point $q^U \in C^U$ such that $g(q^U)=q^W$. 
	By construction of $C ^W$ and $C^U$, we know that there exists a point $p\in C$ such that $$q^W=\sum\limits_{i=1}^N\varphi_i^V(p)w_i$$
	 and a point $p'\in C$ such that 
	 $$q^U=\sum\limits_{i=1}^N\varphi_i^V(p')u_i$$
	 Since we know that cages $W$ and $U$ are similar, we have that, by definition \ref{def:cage_similarity}, there exists a similarity $f$ that maps cage $U$ to $W$ (i.e $w_i=f(u_i)$ for all $i \in \{1,2, \dots, N\}$). It turns out that with $g=f$ and $p^U=p^W$ define the similarity between contours:
	 $$q^W=\sum\limits_{i=1}^N\varphi_i^V(p)w_i=\sum\limits_{i=1}^N\varphi_i^V(p)f(u_i)=f\Big(\sum\limits_{i=1}^N\varphi_i^V(p)u_i\Big)=f(q^U)$$
	 Therefore we have that the same similarity that maps $W$ to $U$, sends their contours to each other rendering them similar.
	
	To prove the second implication, a more elaborate solution is required. We only need to prove this in the case of if $U$ is the shifted cage of $W$ since if we have that, any similar cage would only imply a similarity function.
	To see that a cage and its shifted cage produces a similar curves, let us take two cages $W^0=(w_1, w_2, \dots, w_3)$  and one of its shifted (we take the shift k=1 for simplicity) $W^1=(w_1^1,w_2^1,\dots, w_N^1)=(w_2,w_3, \dots, w_N, w_1)$.
	
	If we see that their images\footnote{In this context, image refers to the target set of a function.} of $C$, respectively $C^{W^0}$ and $C^{W^1}$ are congruent, that is $C^{W^0}=C^{W^1}$, then they would be similar because the identity function would be the similarity between them.
	
	To see this we have to see if every point $q$ in $C^{W^0}$ is in $C^{W^1}$.
	We have that every point in $C^{W^0}$ can be expressed as 

		$$q=\sum\limits_{j=1}^N\varphi_j^V(p)w_j$$

	where $p \in C$ is in the initial contour. If we can find a point $p^1$ in $C$ such that
	
		$$q=\sum\limits_{j=1}^N\varphi_j^V(p^1)w_j^1$$
	
	it would do.
 
  
  The mean value coordinates of a point $p$ with respect to control point $v_i$ are calculated using the angles $\alpha_1$, $\alpha_2$ with its neighboring control points $v_{i-1}$ and $v_{i+1}$ respectively. In figure \ref{fig:proof1_mvc}, we have an example with the circumference contour $C$ and the cage $V=(v_1,v_2,..,v_N)$  (N=6 in the image). Point $p$ has the mean value coordinates $\varphi^V(p)=(\lambda_1,\lambda_2, \dots, \lambda_N)$. If we apply a rotation $R^1$ of $\alpha_{R^1}=-\frac{2\pi}{N}$ radians and center $p_c$. We have that $R^1(v_i)=v_{i+1}$ and the rotated point $p^1=R^1(p)$ would still be on the contour $C$. Furthermore, It would maintain the distance to the rotated control point $R^1(v_i)=v_{i+1}$, as well as the angles to their rotated points, because of the property of angle invariance through similarities.
  
  Therefore we can say that for every point, $p$, there exists a point $p^1=R^1(p)$ such that, the mean value coordinates are the same but shifted: this can be done for any $R^k(p)=-\frac{2\pi}{N}*k$ for $k\in {1,2, \dots, N}$;
   
  
  $$\varphi^V(R^1(p))= (\lambda_2, \lambda_3, ..., \lambda_N, \lambda_{1}) $$
  $$\varphi^V(R^2(p))= (\lambda_3, \lambda_2, ..., \lambda_{N-1}, \lambda_{2}) $$
  $$\dots$$
  $$\varphi^V(R^k(p))= (\lambda_{k+1}, \lambda_{k+2}, ..., \lambda_{N-k-1}, \lambda_{k}) $$
  $$\dots$$
  $$\varphi^V(R^{N-1}(p))= (\lambda_{N}, \lambda_{1}, ..., \lambda_{N-2}, \lambda_{N-1}) $$
  
    
  So, now that we have these points, we know that given any point $q\in C^{W^0}$, there does exist a point
  $p'^1\in C$ so that $q=\sum\limits_{j}^N\varphi_j^V(p)w_j^2$ and it is in particular $p'=R^1(p)$.
  Since we have the following:
  

	  $$q=\sum\limits_{j}^N\varphi_j^V(p)w_j^1=w_1\lambda_1+ w_2\lambda_2+ \dots + w_N\lambda_N =
	  w_2\lambda_2+ w_3\lambda_3+\dots+w_N\lambda_N+w_1\lambda_1= \sum\limits_{j}^N\varphi_j^V(p')w_j^1$$

  Since we can generalize for any shift $k\in \{1,2,\dots,N\}$ with rotation $R^k$, the proposition is proven.
  \end{proof}

  Furthermore, it would be interesting to prove the opposite implication thereby creating a class of equivalence between shapes in the contour family $\mathcal{F}_C^V$ defined by the cage $V$ and initial contour $C$ from the initial configuration. This will be left for future work.
  
   
   \begin{figure}[h!]
   	\centering
   	{\includegraphics[width=\textwidth]{images/fig_proof1_mvc.png}}
   	\caption{Illustration of the existence of a point $p^1$ needed to prove the second implication in proposition \ref{prop:prop1}}
   	\label{fig:proof1_mvc}
   \end{figure}
   
 

\section{Shape description}
\label{subsec:shape_description}

One of the challenges seen especially in medical imaging is that it is often hard to find relevant points in a region that might help to determine structure or orientation of an object that apparently has none. These points are commonly called \textit{landmarks} and are used to build the shape models of an object. It is often the case in medical imaging that these points are unseen, latent or that they are simply characterized by their shape. 

Shape description can also be applied to image retrieval. This application supposes huge databases of images which from we want to retrieve a certain subset or a specific image given specific characteristics. For example in medical imaging, if we want to withdraw all the cases of patients with a similarly shaped elements, such as \textit{caudates}, \textit{putamen}, or \textit{nucleous}, a fast search would be needed to compare with all patient files. This search would require two things: invariance in translation, rotation and scale, and that each element in this dataset could be indexed so that fast and effective retrieval and comparison may be applied. 

The literature in shape comparison is a rich and vast field of research~\cite{Bartolini:2005:WAR:1032293.1032578, Abbasi:1999:CSS:323498.323506}. One of the best methods of shape description are Discrete Fourier Transforms (DFT). These provide a description of the curvature of a shape with respect to a variable $t$ which indicates the point in the curve it is in. These are invariant to translation and uniform change in scale, but the starting point of the shape is critical to know whether two shapes~\cite{Bartolini:2005:WAR:1032293.1032578} are similar through a rotation or not. 

Another interesting method is the Curvature Scale-Space (CSS) shape descriptor. This descriptor provides a representation of a contour which represents the time of inflection or union of pairs of points of the shape as it is progressively smoothed~\cite{Abbasi:1999:CSS:323498.323506}. This descriptor also presents the same problem in rotation, where a shift must be applied to find the right starting point. Figure \ref{fig:css_descriptor_im} shows the example of how as the contour of a shark is smoothed, the less relevant points are joined.

  \begin{figure}[h!]
  	\centering
  	{\includegraphics[width=0.7\textwidth]{images/css_descriptor_im.png}}
  	\caption{CSS smoothing process of a shape and decreasing number of the points with curvature change (image from \cite{Abbasi:1999:CSS:323498.323506}).} 	
  	\label{fig:css_descriptor_im}
  \end{figure}

Both of these methods, which are the most used in this field~\cite{5963789,Zhang200339} provide very good solutions to indexing, description and even rotation~\cite{Zhang200339}. Similarly, Through a segmentation with CAC, once we fix a regular initial cage-contour configuration with $N$ points (defined in \ref{def:initial_config}), we have an approximated shape of the curve which can be described exactly with the $N$ points in the segmented cage, making it a good index as well as a useful descriptor. Also, as we have proved in proposition \ref{prop:prop1}, that similar cages with similar shifted cages represent the same contour. This is useful for fast comparison since we only require at most $N$ comparisons to see the invariance through rotation.

However, CAC offers an advantage over these methods in being a segmentation method, while the previous shape descriptors require a segmentation step which provides a simple connect region. We obtain both in a single process and thanks to the cage parametrization, the segmentation can be corrected by a user by moving the points in the cage.

 
\section{Image Morphing and Warping}
\label{subsec:morphing_warping}


 Image morphing is the interpolation between two \textit{images} while warping is the deformation of the shape of an image. We are interested in morphing objects into each other. 
 
 If we have a regular cage-contour configuration, $(C, V, r)$, and we segment two objects $O_1$  and $O_2$ in images $I^1$ and $I^2$ respectively, then 
 
 \begin{enumerate}
 	\item By proposition \ref{prop:prop1}, if the resulting cages $V^1$ and $V^2$ are similar or similar to a shifted cage, the contours are similar.\label{point:point1}
 	\item By property \ref{property:mvc2}, if there exists a similarity f between cages, then by that similarity the mean value coordinates of $O_1$ with respect to $V^1$ are equal to the mean value coordinates of $f(O_2)$ with respect to $V^2$.
 	\item In the proof or proposition \ref{prop:prop1}, we show that we can always find a shift of a shifted cage so that we may find the similarity f.
 \end{enumerate}
 
 By these three statements, we have that if the segmentation of $O_1$ and $O_2$ gives two cages $V_1$ and $V_2$ respectively that are similar or shifted of similar cages, the same similarity sends $O_1$ to $O_2$. 
 
 If we want to morph two objects $O_1\in I^1$ and $O_2\in I^2$ which respectively have segmentation $V^1$ and $V^2$, then we have that we can define an intermediate cage:
 \begin{equation}
 V^w=V^1*w + V^2*(1-w)
 \end{equation} 
  where $w\in [0,1]$ such that if two cages are similar, they are also similar to their intermediate. For any two cages in general this interpolation is depicted in figure \ref{fig:intermediate_cage}.
  
  \begin{figure}[h]
  	\centering
  	{\includegraphics[width=0.5\textwidth]{images/qualitative_tests/intermediate_cage.png}}
  	\caption{Intermediate cage in morphing}
  	\label{fig:intermediate_cage}
  \end{figure}
  
 
 In the new interpolated image, we now want to find pixel by pixel its corresponding values in each image, and apply a weighted mean to obtain the interpolated value. since they are similar, we have 
 
 \begin{equation}
	 I^w(p^w) = w*I^1\Bigg(\sum\limits_{i=1}^N\varphi_i(p^w)v_i^1\Bigg)+ (1-w)*I^2\Bigg(\sum\limits_{i=1}^N\varphi_i(p^w)v_i^2\Bigg)
 \end{equation}
 
 Now the problem emerges in practice since it is practically impossible for two cages to be similar after a segmentation however they can be similar to a slight deformation of the cage. Thanks to the smooth properties of the deformation this allows for decent morphing through interpolation of the cages. Figures \ref{fig:car_morphing}  and \ref{fig:car_fruits} are examples we created by interpolation of cages. The former is done automatically by finding the shift of the cages that best corresponds to a similarity using a turning function we implemented, while in the fruit images, we assigned a correspondence in the cages that are not similar to show the smoothness this deformation provides regardless.

  \begin{figure}[h!]
  	\centering
  	{\includegraphics[width=1\textwidth]{images/car_morphing.png}}
  	\caption{Morphing a family car to a sports car automatically through mean value coordinates from a segmentation with CAC  (Initial image from \url{http://www.wellclean.com/wp-content/themes/artgallery_3.0/images/car1.png}, final image from \url{http://www.wellclean.com/wp-content/themes/artgallery_3.0/images/car3.png})}
  	\label{fig:car_morphing}
  \end{figure}
  
  
  \begin{figure}[h!]
  	\centering
  	{\includegraphics[width=1\textwidth]{images/car_fruits.png}}
  	\caption{Morphing from an apple to a pear with a CAC segmentation (Initial and final images from~\cite{Marko2013a})}
  	\label{fig:car_fruits}
  \end{figure}
  
%
%\begin{figure}[h]
%	\centering
%	{\includegraphics[width=0.5\textwidth]{images/cage_morphing.png}}
%	\caption{Different cylindric colors spaces (image from \cite{WinNT}) and the spherical color-space (image from \cite{grf2088}).}
%	\label{fig:cage_morphing}
%\end{figure}




\newpage

\chapter{Conclusions and Future Work}
\label{sec:conclusions}

\noindent

In this thesis we have made various contributions to the object segmentation method Cage Active Contours (CAC) originally introduced in \cite{ipcac2015}. Our contributions include 1) the introduction of two different energy functions on different color spaces which have greatly enhanced the potential of an otherwise limited method. 2) the experimental validation of our improvements with the previous energies in CAC as well as with three different related methods.  3) the formalization, in mathematical terms, of some the implications and uses of the resulting segmentation \textit{cage} (a component of CAC which is used to parametrize the contour). 4)  a highlight of the possible applications of this cage to image morphing, warping as well as in shape description. 5) the creation of a public implementation in Python (with some wrapped functions in C) of cage active contours.

\section{Conclusions}

 One of the main contributions in this work is the creation of two different energy functions. These are enhanced versions of simpler energies proposed so far in CAC which have also been extended to two different color spaces. The first one is the Multivariate Mixture Gaussian energy which is an extension to the RGB color space of the Gaussian energy defined in \cite{ipcac2015} with two more improvements: the ability to capture multiple value components in each region by using Mixture Gaussian density function, and the incorporation of an initial seed which will provide the energy with prior information about the foreground and background's distributions.

The second energy is the Mean Hue Energy, which is the analogous of the Mean Energy in \cite{ipcac2015} but on the Hue component of the HSI/HSV color spaces. Because of the Cyclic nature of the Hue component, we have had to turn to cyclic spaces to use concepts such as distance, directed distance as well as a way of finding the gradient of a of the image with respect to a control point's value.

Through quantitative and qualitative experimentation on three different datasets (section \ref{sec:experiments}), we have observed that the contribution with the most impact are the extension of the Gaussian energy to the RGB color space followed by the ability of Mixture Gaussian energies to describe a region with more than one component thus obtaining a more sophisticated representation of a region, as opposed to the polarization of pixel values in previous energies. We have also seen how the incorporation of a seed in each region does not only simplify the process but it also increases its computational speed. In the case of color images, we have seen how the Mean Hue energy, despite its simplicity, can occasionally outperform the Multivariate Mixture Gaussian energy thanks to its focus on the Hue component of the image, which reflects an intrinsic property of objects, invariant to illumination.

Furthermore, we have mathematically formalized the concepts of \textit{cage}, \textit{contour}, \textit{family of contours} and others to be able to prove that two contours are similar if their cages are similar given some intial conditions. This theoretical proof, along with the properties of mean value coordinates, used widely in computer graphic applications, have allowed us to define the conditions which allow for automatic morphing and warping between similar objects, as well provide some initial intuitions for the possibility of shape description.

Our last contribution is a public implementation of Cage Active Contours in Python with some wrappers in C. The code contains different Energy functions we have presented including the ones presented  in~\cite{ipcac2015}, as well as some tools for automatic morphing and warping. The code can be found in \url{https://github.com/Jeronics/cac-segmenter/}.


\section{Future Work}
In this work, we have also observed and pointed out the limitations that hinder the performance of the segmentation using Cage Active Contours. We have seen that CAC are not designed for high precision segmentation of arbitrary images, but rather, they provide a smooth general contour of the image which can be used for other purposes and applications. The limitations of this method can be divided into two categories: those that are dependent on energy functions and those that are dependent on the cage. 

With these considerations in mind, we have proposed a set of solutions which could be evaluated and considered for future work.

The first point to address is the cage. We have seen that the restrictions of the first stage on the vertex movements are too strong and prevent CAC from adopting complex figures. Also, they prevent the cage from rotate to better adapt to a shape. Possible solution include the weighted mean between the real direction and the projected direction or  the exploration of other internal energies that could be applied to the cage for stability.

In terms of energy functions, the most challenging task ahead, by far, is to improve the Mean Hue energy which, despite its theoretically good properties in illumination invariance, has performed poorly on the quantitative experiments on real images. However we are confident that through some improvements we could be able to rise its performance to a competent level. The first improvement we propose is the adaptation of the Gaussian Energy to this space and ultimately to the Mixture Gaussian model. This extension requires the use of analogous density functions in the cyclic domain which are known as Wrapped Density Functions. The second improvement would come from the extension of an energy to the whole HSI or HSV space. As we have concluded in section \ref{subsec:extending_to_hsi} this requires tools from directional statistics and in particular, the study of cylinder spaces which these color spaces define.

As far as the implementation goes, it would be interesting to allow for a more intuitive and user-friendly interactive interface. Some of the features this could include:
\begin{enumerate}
	\item User interaction with the cage so that control points would be able to be dragged to provide a better initialization, and to correct a segmentation at a certain point.
	\item A Morphing and Warping interface that would allow for fine-tuning of morphed objects or reassigning of correspondence in points as we have done in the morphing between the pear and apple in figure~\ref{fig:car_fruits}.
\end{enumerate} 

An interesting point we have begun to raise in this thesis is the formalization in mathematical terms of the different components in Cage Active Contours to study their properties and limitations in different applications once a segmentation has been achieved. In the process of bringing solutions, more questions are brought forward to discuss and tackle in future work. Namely: \\
Given a regular cage-configuration condition:
\begin{itemize}
	\item What is the complexity of contours that can be expressed in a contour Family?
	\item Which characteristics must a cage fulfill in order to avoid its corresponding contour to self intersect?
	\item Is there a bijective map between the equivalence classes of cages and the equivalence classes of contours in a contour Family?
\end{itemize}




\vfill
\renewcommand{\bibname}{\refname} % To change the heading of the chapter from Bibliography to References
\addcontentsline{toc}{chapter}{\bibname}

\bibliographystyle{apalike}

\bibliography{example.bib}

%\section*{\uppercase{Appendix}}
%
%\noindent If any, the appendix should appear directly after the
%references without numbering, and not on a new page. To do so please use the following command:
%\textit{$\backslash$section*\{APPENDIX\}}

\vfill
\end{document}